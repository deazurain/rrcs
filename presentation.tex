\documentclass{beamer}
\usepackage{etex}
\usepackage[latin1]{inputenc}
\usepackage[dutch]{babel}

\usepackage{amssymb}
\usepackage{amsmath}
\usepackage[numbers]{natbib}
\usepackage{xcolor}
% \usepackage[colorlinks,linkcolor=blue,urlcolor=blue,citecolor=blue]{hyperref}
\usepackage{mathrsfs}
\usepackage{comment}

\usepackage{tabularx}
\usepackage{booktabs}
\usepackage{amsthm}
\usepackage{nameref}
\usepackage{wrapfig}
\usepackage{pgf}
\usepackage{algorithm2e}
\usepackage{pgfpages}

\theoremstyle{definition}
\providecommand{\SetAlgoLined}{\SetLine}
\providecommand{\DontPrintSemicolon}{\dontprintsemicolon}
\newcommand{\inputtikz}[1]{\input{tikz/#1}}
\inputtikz{preamble}
% long footnote section fixes
\dimen\footins=40\baselineskip\relax
\raggedbottom
\addtolength{\topskip}{0pt plus 10pt}
\interfootnotelinepenalty=10000
\newcommand{\TODO}[1]{{\color{red}\textbf{TODO: #1}}}
\newcommand{\mick}[1]{{\color{red}\emph{#1}}}
% \newtheorem{example}{Example}[section]
\newcommand{\res}[0]{\ensuremath{R}} %resources
\newcommand{\av}[2]{\ensuremath{av(r_{#1}, t_{#2})}} %availability of resource #1 at time #2
\newcommand{\capa}[1]{\ensuremath{cap(r_{#1})}} %capacity
\newcommand{\dur}[1]{\ensuremath{dur(v_{#1})}} %durability
\newcommand{\usage}[2]{\ensuremath{usage(v_{#1}, r_{#2})}} %usage of resource #2 by activity #1
\newcommand{\start}[1]{\ensuremath{start(v_{#1})}} %start time
\newcommand{\makespan}[1]{\ensuremath{C_{max}(#1)}} %makespan
\newcommand{\mindelay}[2]{\ensuremath{delay_{min}(t_{#1}, t_{#2})}} %minimum delay
\newcommand{\maxdelay}[2]{\ensuremath{delay_{max}(t_{#1}, t_{#2})}} %maximum delay
\newcommand{\weight}[2]{\ensuremath{weight(t_{#1}, t_{#2})}} %weight
% \newenvironment{definition}[1][Definition]{\begin{trivlist}
% \item[\hskip \labelsep {\bfseries #1}]}{\end{trivlist}}

%\item<2-> laat dat item pas in deel 2 van een slide zien \item<3-6> tussen 3 en 6 etc
%\section[Kort]{Lange naam} zorgt ervoor dat in de bovenbalk de korte gebruikt wordt en op de slides en toc de lange

% presentation theme
\usetheme{Szeged}

% lose the navigation buttons
\setbeamertemplate{navigation symbols}{}

% insert a local toc for each section
\AtBeginSection[] {
    \begin{frame}<beamer>
        \frametitle{\insertsection~overzicht}

        \tableofcontents[currentsection,sectionstyle=hide/hide,subsectionstyle=show/show/hide]
    \end{frame}
}

% number figures
\setbeamertemplate{caption}[numbered]

% notes
\setbeameroption{show notes}
\setbeameroption{show notes on second screen=right}

\title[RCPS]{Resource Constrained Project Scheduling\\
An Approach}
\author{M. van Gelderen  \and
    R.M. de Lange \and
    B. Gris\`el \and
    F. van Tienen}
\institute{TU Delft}
\date{\today}

\begin{document}

\begin{frame}
\titlepage
\end{frame}

\section*{Intro}
\begin{frame}
  \frametitle{Inleiding}
  \begin{itemize}
		\item Veel planningsproblemen in dagelijks leven
		\note{bla }
		\item Kleine schaal
		\item Grote schaal
		\item Resource Constrained Project Scheduling Problem
  \end{itemize}
\end{frame}

\begin{frame}
    \frametitle{Inhoud}
    \tableofcontents
\end{frame}

% Scenario (intuitief, plaatje)
% Scenario (formeler, en wat willen we bereiken)
% Hoe gaan we dit oplossen?
% Om ons heen kijken => er zijn al mensen die een tijdsprobleem hebben opgelost in poly-tijd
% Laten we eens kijken of we dat kunnen gebruiken (hoe werkt zo'n STN)
% Ja want precedence constraints zijn om te schrijven naar STN constraints
% Er is een algoritme om dit op te lossen => oplossing laten zien
% Uit die oplossing een schema halen

\section{Scenario}

\begin{frame}
    \frametitle{Scenario}
    PLAATJE VAN HUIS MET HEK EN DEUR HIER
    
    \begin{center}
    	De deur, het hek en buitenmuren moeten worden geschilderd
    \end{center}
\end{frame}

\begin{frame}
    \begin{center}
    	Hoe doen we dit zo effici\"{e}nt mogelijk?
    \end{center}
\end{frame}

\subsection{Taken}
\begin{frame}
    \frametitle{Taken}
    Wat er allemaal moet gebeuren:
     \begin{itemize}
    	\item Deur schilderen
	\item Hek schilderen
	\item Buitenmuren schilderen
	\item<2-> Deur schuren
	\item<2-> Hek schuren
	\item<3-> Materiaal schoonmaken
    \end{itemize}
\end{frame}

\begin{frame}
    \frametitle{Taken}
    Wat er allemaal moet gebeuren:
     \begin{itemize}
    	\item Deur schilderen (2 uur)
	\item Hek schilderen (1 uur)
	\item Buitenmuren schilderen (5 uur)
	\item Deur schuren (3 uur)
	\item Hek schuren (2 uur)
	\item Materiaal schoonmaken (1 uur)
    \end{itemize}
\end{frame}

\begin{frame}
    	\frametitle{Taken}
   	Er zijn een aantal voorwaarden:
	\begin{enumerate}
	    	\item Een oppervlak moet eerst geschuurd worden voordat het kan worden geschilderd
		\item Materiaal schoonmaken doe je als laatste
	\end{enumerate}
\end{frame}

\subsection{Taken en Relaties}
\begin{frame}
	\frametitle{Taken en Relaties}
	\vspace{-1em}
	\begin{figure}[ht]
		\makebox[\textwidth][c]{\resizebox{.8\paperwidth}{!}{
			\inputtikz{precendence_graph}
		}}
		\vspace{-1em}
	\end{figure}
\end{frame}

\subsection{Doel}
\begin{frame}
    	\frametitle{Doel}
   	\textbf{Gegeven}
	\begin{enumerate}
		\item Een set $V$ van taken $\{v_1,\dots,v_n\}$ met een bepaalde tijdsduur
		\item Een set $E$ van precedence constraints met elementen $(v_i, v_j)$ zodanig dat $v_i, v_j \in V$ en $v_i$ voor $v_j$ moet worden uitgevoerd
	\end{enumerate}
	
	\textbf{Gevraagd}\\
	Een schema waarin voor elke activiteit in $V$ een starttijd is toegewezen zodanig dat aan alle precedence relaties in $E$ voldaan wordt.
\end{frame}

\begin{frame}
	\begin{center}
		Is er een probleem wat hier op lijkt? Heeft iemand een soortgelijk probleem al opgelost?
	\end{center}
\end{frame}

\begin{frame}
    	\frametitle{Simple Temporal Problem}
	
	\begin{itemize}
		\item Omstreeks 1990 gepubliceerd
		\item Is oplosbaar in polynomiale tijd
	\end{itemize}
   	
\end{frame}

\begin{frame}
    	\frametitle{Hoe werkt het}
	
	Precendence Graph (met bolletjes als nodes + [0,inf])
	\begin{figure}[ht]
		\inputtikz{stn_precedence}
		\vspace{-1em}
	\end{figure}
\end{frame}

\begin{frame}
    	\frametitle{Hoe werkt het}
	
	Precendence Graph (met bolletjes als nodes + [0,inf])
	\begin{figure}[ht]
		\inputtikz{stn_weighted}
		\vspace{-1em}
	\end{figure}
\end{frame}

\begin{frame}
    	\frametitle{Hoe werkt het}
	
	Precendence Graph (met twee bolletjes als nodes + [0,inf])
	\begin{figure}[ht]
		\inputtikz{stn_expanded}
		\vspace{-1em}
	\end{figure}
\end{frame}

\begin{frame}
    	\frametitle{Hoe werkt het}
	
	Precendence Graph (met bolletjes als nodes + [0,inf] + activiteiten uitsplitsen)
	\begin{figure}[ht]
		\inputtikz{stn}
		\vspace{-1em}
	\end{figure}
\end{frame}

\begin{frame}
    	\frametitle{Hoe werkt het}
	
	Precendence Graph (met bolletjes als nodes + [0,inf] + activiteiten uitsplitsen + dummy)
	\begin{figure}[ht]
		\inputtikz{stn_dummy}
		\vspace{-1em}
	\end{figure}
\end{frame}

\begin{frame}
    	\frametitle{Hoe werkt het}
	
	STN met de tight boundaries
\end{frame}

\begin{frame}
\frametitle{Schedule}
	\vspace{-1.2em}
	\begin{figure}[ht]
		\makebox[\textwidth][c]{\resizebox{.36\paperwidth}{!}{
			\inputtikz{schedule_infeasible_colored_profile}
		}}
		\vspace{-1.3em}
		\caption{Activity graph}
		%\label{fig:activity_graph}
	\end{figure}
\end{frame}

\section{Probleem}
\begin{frame}
	\begin{itemize}
		\item Schema werkt, wat gaat er dan toch fout?
		\item Middelen, of \emph{resources}
		\item De schilders hebben alleen:
		\item 1 schuurmachine
		\item 2 verfkwasten
	\end{itemize}
\end{frame}

\begin{frame}
	\begin{itemize}
		\item Schema werkt, wat gaat er dan toch fout?
		\item Middelen, of \emph{resources}
		\item De schilders hebben alleen:
		\item 1 schuurmachine
		\item 2 verfkwasten
	\end{itemize}
\end{frame}

% explain the terminology and concepts of RCPSP
\section{RCPSP}

% explain resource constraints
\subsection{Resource constraints}
\begin{frame}
	\frametitle{Resource constraints}
	\begin{itemize}
		\item Explain what a resource is
		\item Give painting brush example
		\item introduce resource set notation
		\item Explain what capacity is
		\item \mick{notation not necessary imo}
		\item Briefly mention the term usage (the concept of usage should already be clear)
		\item Make transition to the usage diagram, explain that there are more that the two mentioned activities in the project
	\end{itemize}
\end{frame}
	
% demonstrate how resources are linked to activities
\begin{frame}
	\frametitle{Resource usage}
	\vspace{-.2em}
	\begin{figure}[ht]
		\makebox[\textwidth][c]{\resizebox{.7\paperwidth}{!}{
			\inputtikz{usage}
		}}
		\vspace{-1em}
		\caption{The relation between activities and resources}
		\label{fig:activity_graph}
	\end{figure}
\end{frame}

% explain schedules
\subsection{Schedules}
\begin{frame}
	\frametitle{Schedules}
	\begin{itemize}
		\item What is a schedule
		\item What is feasibility
	\end{itemize}
\end{frame}

\begin{frame}
	\frametitle{Feasible schedule}
	\vspace{-1em}
	\begin{figure}[ht]
		\makebox[\textwidth][c]{\resizebox{.38\paperwidth}{!}{
			\inputtikz{schedule_feasible_profile}
		}}
		\vspace{-1.2em}
		\caption{The relation between activities and resources}
		\label{fig:activity_graph}
	\end{figure}
\end{frame}

\subsection{Summary}

\end{document}