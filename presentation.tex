\documentclass{beamer}
\usepackage[latin1]{inputenc}
\usepackage[dutch]{babel}

\usepackage{amssymb}
\usepackage{amsmath}
\usepackage[numbers]{natbib}
\usepackage{xcolor}
% \usepackage[colorlinks,linkcolor=blue,urlcolor=blue,citecolor=blue]{hyperref}
\usepackage{mathrsfs}
\usepackage{comment}

\usepackage{tabularx}
\usepackage{booktabs}
\usepackage{amsthm}
\usepackage{nameref}
\usepackage{wrapfig}
\usepackage{pgf}
\usepackage{algorithm2e}
\usepackage{pgfpages}

\theoremstyle{definition}
\providecommand{\SetAlgoLined}{\SetLine}
\providecommand{\DontPrintSemicolon}{\dontprintsemicolon}
\newcommand{\inputtikz}[1]{\input{tikz/#1}}
\inputtikz{preamble}
% long footnote section fixes
\dimen\footins=40\baselineskip\relax
\raggedbottom
\addtolength{\topskip}{0pt plus 10pt}
\interfootnotelinepenalty=10000
\newcommand{\TODO}[1]{{\color{red}\textbf{TODO: #1}}}
\newcommand{\mick}[1]{{\color{red}\emph{#1}}}
% \newtheorem{example}{Example}[section]
\newcommand{\res}[0]{\ensuremath{R}} %resources
\newcommand{\av}[2]{\ensuremath{av(r_{#1}, t_{#2})}} %availability of resource #1 at time #2
\newcommand{\capa}[1]{\ensuremath{cap(r_{#1})}} %capacity
\newcommand{\dur}[1]{\ensuremath{dur(v_{#1})}} %durability
\newcommand{\usage}[2]{\ensuremath{usage(v_{#1}, r_{#2})}} %usage of resource #2 by activity #1
\newcommand{\start}[1]{\ensuremath{start(v_{#1})}} %start time
\newcommand{\makespan}[1]{\ensuremath{C_{max}(#1)}} %makespan
\newcommand{\mindelay}[2]{\ensuremath{delay_{min}(t_{#1}, t_{#2})}} %minimum delay
\newcommand{\maxdelay}[2]{\ensuremath{delay_{max}(t_{#1}, t_{#2})}} %maximum delay
\newcommand{\weight}[2]{\ensuremath{weight(t_{#1}, t_{#2})}} %weight
% \newenvironment{definition}[1][Definition]{\begin{trivlist}
% \item[\hskip \labelsep {\bfseries #1}]}{\end{trivlist}}

%\item<2-> laat dat item pas in deel 2 van een slide zien \item<3-6> tussen 3 en 6 etc
%\section[Kort]{Lange naam} zorgt ervoor dat in de bovenbalk de korte gebruikt wordt en op de slides en toc de lange

% presentation theme
\usetheme{Szeged}

% lose the navigation buttons
\setbeamertemplate{navigation symbols}{}

% insert a local toc for each section
\AtBeginSection[] {
    \begin{frame}<beamer>
        \frametitle{\insertsection~overzicht}

        \tableofcontents[currentsection,sectionstyle=hide/hide,subsectionstyle=show/show/hide]
    \end{frame}
}

% number figures
\setbeamertemplate{caption}[numbered]

% notes
\setbeameroption{show notes}
\setbeameroption{show notes on second screen=right}

\title[RCPS]{Resource Constrained Project Scheduling\\
An Approach}
\author{M. van Gelderen  \and
    R.M. de Lange \and
    B. Gris\`el \and
    F. van Tienen}
\institute{TU Delft}
\date{\today}

\begin{document}

\begin{frame}
\titlepage
\end{frame}

% Scenario (intuitief, plaatje)
% Scenario (formeler, en wat willen we bereiken)
% Hoe gaan we dit oplossen?
% Om ons heen kijken => er zijn al mensen die een tijdsprobleem hebben opgelost in poly-tijd
% Laten we eens kijken of we dat kunnen gebruiken (hoe werkt zo'n STN)
% Ja want precedence constraints zijn om te schrijven naar STN constraints
% Er is een algoritme om dit op te lossen => oplossing laten zien
% Uit die oplossing een schema halen

\section*{Scenario}

\begin{frame}
    \frametitle{Scenario}
    PLAATJE VAN HUIS MET HEK EN DEUR HIER
    
    \begin{center}
    	De deur, het hek en buitenmuren moeten worden geschilderd
    \end{center}
\end{frame}

\begin{frame}
    \begin{center}
    	Hoe doen we dit zo effici\"{e}nt mogelijk?
    \end{center}
\end{frame}

\begin{frame}
    \frametitle{Taken}
    Wat er allemaal moet gebeuren:
     \begin{itemize}
    	\item Deur schilderen
	\item Hek schilderen
	\item Buitenmuren schilderen
	\item<2-> Deur schuren
	\item<2-> Hek schuren
	\item<3-> Materiaal schoonmaken
    \end{itemize}
\end{frame}

\begin{frame}
    \frametitle{Taken}
    Wat er allemaal moet gebeuren:
     \begin{itemize}
    	\item Deur schilderen (2 uur)
	\item Hek schilderen (1 uur)
	\item Buitenmuren schilderen (5 uur)
	\item Deur schuren (3 uur)
	\item Hek schuren (2 uur)
	\item Opruimen (1 uur)
    \end{itemize}
\end{frame}

\begin{frame}
    	\frametitle{Taken}
   	Er zijn een aantal voorwaarden:
	\begin{enumerate}
	    	\item Een oppervlak moet eerst geschuurd worden voordat het kan worden geschilderd
		\item Opruimen doe je als laatste
	\end{enumerate}
\end{frame}

\begin{frame}
	\frametitle{Taken en Relaties}
	\vspace{-1em}
	\begin{figure}[ht]
		\makebox[\textwidth][c]{\resizebox{.8\paperwidth}{!}{
			\inputtikz{activity_graph_simple}
		}}
		\vspace{-1em}
	\end{figure}
\end{frame}

\section*{Intro}
\begin{frame}
  \frametitle{Inleiding}
  \begin{itemize}
		\item Veel planningsproblemen in dagelijks leven
		\note{bla }
		\item Kleine schaal
		\item Grote schaal
		\item Resource Constrained Project Scheduling Problem
  \end{itemize}
\end{frame}

\begin{frame}
    \frametitle{Inhoud}
    \tableofcontents
\end{frame}

\section[Schildersproject]{Voorbeeld: schildersproject}

\subsection{Taken}

\begin{frame}
	\begin{itemize}
	
		\item Deur schuren
		\item Hek schuren
		\item<2-> Muur verven
		\item<2-> Deur verven
		\item<2-> Hek verven
		\item<3-> Opruimen
	\end{itemize}
\end{frame}

\begin{frame}
	\begin{itemize}
		\item Deur schuren (3 uur)
		\item Hek schuren (2 uur)
		\item<2-> Muur verven (5 uur)
		\item<2-> Deur verven (2 uur)
		\item<2-> Hek verven (1 uur)
		\item<3-> Opruimen (1 uur)
	\end{itemize}
\end{frame}

\begin{frame}
	\begin{itemize}
		\item Taken zijn de onderdelen van het project
		\item Deze taken moet gepland worden
		\item Taken hebben een bepaalde duur
	\end{itemize}
\end{frame}

\subsection{Volgorde}
\begin{frame}
	\begin{itemize}
		\item Eerst schuren, daarna verven
		\item Eerst alle andere taken, daarna opruimen
		\item Volgorde is van belang
		\item In RCPSP heet dat: \emph{precedence constraints}
	\end{itemize}
\end{frame}

% demonstrate precedence constraints through graph
\begin{frame}
	\frametitle{Activity graph}
	\vspace{-1em}
	\begin{figure}[ht]
		\makebox[\textwidth][c]{\resizebox{.8\paperwidth}{!}{
			\inputtikz{activity_graph_simple}
		}}
		\vspace{-1em}
		\caption{Activity graph}
		\label{fig:activity_graph}
	\end{figure}
\end{frame}

\subsection{Schema}
\begin{frame}
	\begin{itemize}
		\begin{figure}[ht]
			\centering
			% Author: Mick van Gelderen
% Created from http://www.texample.net/tikz/examples/class-diagram/ and other examples and resources. 

%\documentclass{standalone}
%\usepackage{tikz}
\usetikzlibrary{calc,positioning,shapes,arrows,decorations.pathreplacing}
\usetikzlibrary{shapes.multipart}


%\begin{document}

% This file is included by preamble.tex from base directory ../

\pgfdeclarepatternformonly{swnestripes} {
	\pgfpoint{0cm}{0cm}
}{
	\pgfpoint{2cm}{2cm}
}{
	\pgfpoint{1cm}{1cm}
}{
		\pgfsetlinewidth{10pt}
		\pgfpathmoveto{\pgfpoint{0cm}{0cm}}
		\pgfpathlineto{\pgfpoint{2cm}{2cm}}
		\pgfusepath{stroke}
}

% SICKE SHIT YEAA!
\pgfdeclarepatterninherentlycolored{sandingpainting} {
	\pgfpoint{0cm}{0cm}
}{
	\pgfpoint{4cm}{4cm}
}{
	\pgfpoint{2cm}{2cm}
}{
		\pgfsetlinewidth{10pt}
		\pgfsetcolor{orange!28}
		\pgfpathmoveto{\pgfpoint{0cm}{0cm}}
		\pgfpathlineto{\pgfpoint{10cm}{10cm}}
		\pgfpathclose%
		\pgfusepath{stroke}
		\pgfsetcolor{blue!14}
		\pgfpathmoveto{\pgfpoint{1cm}{0cm}}
		\pgfpathlineto{\pgfpoint{11cm}{10cm}}
		\pgfpathclose%
		\pgfusepath{stroke}
}

\tikzstyle{activity}=[rectangle, draw=black, anchor=south west, minimum height=1cm, minimum width=1cm]
\tikzstyle{sanding}=[pattern=swnestripes, pattern color=orange!28]
\tikzstyle{painting}=[pattern=swnestripes, pattern color=blue!14]
\tikzstyle{sandingpainting}=[pattern=sandingpainting]

\tikzstyle{invalid}=[draw=red, ultra thick]
\tikzstyle{axis}=[->,thick,>=stealth,shorten >=-.3cm,shorten <=-.3cm]
\tikzstyle{yaxisnode}=[text depth=0pt, text height=2ex, rotate=90, yshift=3.7ex, anchor=north east]
\tikzstyle{capacity}=[dashed,shorten >=-.3cm, shorten <=-.3cm]
\begin{tikzpicture}[node distance=1, scale=1, every node/.style={transform shape}]
	\begin{scope}[shift={(0,0)}]
  \draw[axis] (0,0) -- (6,0);% node[anchor=north, yshift=-1em] {Time};
	\draw[axis] (0,0) -- (0,3) node[yaxisnode] {Activities};
	
\node (v1) [activity, sanding, minimum width=3cm] at (0,2) {$v_1$};
\node (v2) [activity, sanding, minimum width=2cm] at (0,1) {$v_2$};
\node (v3) [activity, painting, minimum width=2cm] at (3,2) {$v_3$};
\node (v4) [activity, painting, minimum width=1cm] at (2,1) {$v_4$};
\node (v5) [activity, painting, minimum width=5cm] at (0,0) {$v_5$};
\node (v6) [activity, sandingpainting, minimum width=1cm] at (5,0) {$v_6$};

\draw[dotted] (v2.south east) -- ++(0,-1) node[anchor=north] {$t_2$};
\draw[dotted] (v4.south east) -- ++(0,-1) node[anchor=north] {$t_3$};
\node [anchor=north west] at (0,0) {$t_0$};
\node [anchor=north] at (5,0) {$t_5$};
\node [anchor=north] at (6,0) {$t_6$};
\end{scope}

\begin{scope}[shift={(0,-3)}]
  \draw[axis] (0,0) -- (6,0);% node[anchor=north] {Time};
	\draw[capacity, shorten <=0cm] (0,1) -- (6,1);
  \draw[axis] (0,0) -- (0,2) node[yaxisnode] {$r_1$ usage};

  \node (v1) [activity, sanding, minimum width=3cm] at (0,0) {$v_1$};
  \node (v2) [activity, sanding, invalid, minimum width=2cm] at (0,1) {$v_2$};
	\node (v6) [activity, sandingpainting, minimum width=1cm] at (5,0) {$v_6$};
	
  \draw[dotted] (v2.south east) -- ++(0,-1) node[anchor=north] {$t_2$};
  \node [anchor=north west] at (0,0) {$t_0$};
  \node [anchor=north] at (3,0) {$t_3$};
  \node [anchor=north] at (5,0) {$t_5$};
	\node [anchor=north] at (6,0) {$t_6$};
\end{scope}
  
  
\begin{scope}[shift={(0,-6)}]
  \draw[axis] (0,0) -- (6,0);% node[anchor=north, yshift=-1em] {Time};
	\draw[capacity] (0,2) -- (6,2);
  \draw[axis] (0,0) -- (0,2) node[yaxisnode] {$r_2$ usage};

  \node (v3) [activity, painting, minimum width=2cm] at (3,1) {$v_3$};
  \node (v4) [activity, painting, minimum width=1cm] at (2,1) {$v_4$};
  \node (v5) [activity, painting, minimum width=5cm] at (0,0) {$v_5$};
  \node (v6) [activity, sandingpainting, minimum width=1cm] at (5,0) {$v_6$};

  \draw[dotted] (v4.south west) -- ++(0,-1) node[anchor=north] {$t_2$};
  \draw[dotted] (v3.south west) -- ++(0,-1) node[anchor=north] {$t_3$};
  \node [anchor=north west] at (0,0) {$t_0$};
  \node [anchor=north] at (5,0) {$t_5$};
  \node [anchor=north] at (6,0) {$t_6$};
\end{scope}
\end{tikzpicture}

%\end{document}
			\caption{Een schema voor het schildersproject}
			\label{fig:activity_graph}
		\end{figure}
	\end{itemize}
\end{frame}

\section{Probleem}
\begin{frame}
	\begin{itemize}
		\item Schema werkt, wat gaat er dan toch fout?
		\item Middelen, of \emph{resources}
		\item De schilders hebben alleen:
		\item 1 schuurmachine
		\item 2 verfkwasten
	\end{itemize}
\end{frame}

\begin{frame}
	\begin{itemize}
		\item Schema werkt, wat gaat er dan toch fout?
		\item Middelen, of \emph{resources}
		\item De schilders hebben alleen:
		\item 1 schuurmachine
		\item 2 verfkwasten
	\end{itemize}
\end{frame}

% explain the terminology and concepts of RCPSP
\section{RCPSP}

% explain resource constraints
\subsection{Resource constraints}
\begin{frame}
	\frametitle{Resource constraints}
	\begin{itemize}
		\item Explain what a resource is
		\item Give painting brush example
		\item introduce resource set notation
		\item Explain what capacity is
		\item \mick{notation not necessary imo}
		\item Briefly mention the term usage (the concept of usage should already be clear)
		\item Make transition to the usage diagram, explain that there are more that the two mentioned activities in the project
	\end{itemize}
\end{frame}
	
% demonstrate how resources are linked to activities
\begin{frame}
	\frametitle{Resource usage}
	\vspace{-.2em}
	\begin{figure}[ht]
		\makebox[\textwidth][c]{\resizebox{.7\paperwidth}{!}{
			\inputtikz{usage}
		}}
		\vspace{-1em}
		\caption{The relation between activities and resources}
		\label{fig:activity_graph}
	\end{figure}
\end{frame}

% explain schedules
\subsection{Schedules}
\begin{frame}
	\frametitle{Schedules}
	\begin{itemize}
		\item What is a schedule
		\item What is feasibility
	\end{itemize}
\end{frame}

\begin{frame}
	\frametitle{Feasible schedule}
	\vspace{-1em}
	\begin{figure}[ht]
		\makebox[\textwidth][c]{\resizebox{.38\paperwidth}{!}{
			\inputtikz{schedule_feasible_profile}
		}}
		\vspace{-1.2em}
		\caption{The relation between activities and resources}
		\label{fig:activity_graph}
	\end{figure}
\end{frame}

\subsection{Summary}

\end{document}