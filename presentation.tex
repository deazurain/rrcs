\documentclass{beamer}
\usepackage[latin1]{inputenc}
\usepackage[dutch]{babel}

\usepackage{amssymb}
\usepackage{amsmath}
\usepackage[numbers]{natbib}
\usepackage{xcolor}
% \usepackage[colorlinks,linkcolor=blue,urlcolor=blue,citecolor=blue]{hyperref}
\usepackage{mathrsfs}
\usepackage{comment}

\usepackage{tabularx}
\usepackage{booktabs}
\usepackage{amsthm}
\usepackage{nameref}
\usepackage{wrapfig}
\usepackage{pgf}
\usepackage{algorithm2e}

\theoremstyle{definition}
\providecommand{\SetAlgoLined}{\SetLine}
\providecommand{\DontPrintSemicolon}{\dontprintsemicolon}
\newcommand{\inputtikz}[1]{\input{tikz/#1}}
\inputtikz{preamble}
% long footnote section fixes
\dimen\footins=40\baselineskip\relax
\raggedbottom
\addtolength{\topskip}{0pt plus 10pt}
\interfootnotelinepenalty=10000
\newcommand{\TODO}[1]{{\color{red}\textbf{TODO: #1}}}
\newcommand{\mick}[1]{{\color{red}\emph{#1}}}
% \newtheorem{example}{Example}[section]
\newcommand{\res}[0]{\ensuremath{R}} %resources
\newcommand{\av}[2]{\ensuremath{av(r_{#1}, t_{#2})}} %availability of resource #1 at time #2
\newcommand{\capa}[1]{\ensuremath{cap(r_{#1})}} %capacity
\newcommand{\dur}[1]{\ensuremath{dur(v_{#1})}} %durability
\newcommand{\usage}[2]{\ensuremath{usage(v_{#1}, r_{#2})}} %usage of resource #2 by activity #1
\newcommand{\start}[1]{\ensuremath{start(v_{#1})}} %start time
\newcommand{\makespan}[1]{\ensuremath{C_{max}(#1)}} %makespan
\newcommand{\mindelay}[2]{\ensuremath{delay_{min}(t_{#1}, t_{#2})}} %minimum delay
\newcommand{\maxdelay}[2]{\ensuremath{delay_{max}(t_{#1}, t_{#2})}} %maximum delay
\newcommand{\weight}[2]{\ensuremath{weight(t_{#1}, t_{#2})}} %weight
% \newenvironment{definition}[1][Definition]{\begin{trivlist}
% \item[\hskip \labelsep {\bfseries #1}]}{\end{trivlist}}

% presentation theme
\usetheme{Szeged}

% lose the navigation buttons
\setbeamertemplate{navigation symbols}{}

% insert a local toc for each section
\AtBeginSection[] {
    \begin{frame}<beamer>
        \frametitle{\insertsection~overzicht}

        \tableofcontents[currentsection,sectionstyle=hide/hide,subsectionstyle=show/show/hide]
    \end{frame}
}

% number figures
\setbeamertemplate{caption}[numbered]

\title[RCPS]{Resource Constrained Project Scheduling\\
An Approach}
\author{M. van Gelderen  \and
    R.M. de Lange \and
    B. Gris\`el \and
    F. van Tienen}
\institute{TU Delft}
\date{\today}

\begin{document}

\begin{frame}
\titlepage
\end{frame}

\section*{Intro}
\begin{frame}
    \frametitle{Inleiding}
    \begin{itemize}
	\item Veel planningsproblemen in dagelijks leven
        \item Kleine schaal
        \item Grote schaal
	\item Resource Constrained Project Scheduling Problem
    \end{itemize}
\end{frame}

\begin{frame}
    \frametitle{Inhoud}
    \tableofcontents
\end{frame}

\section[Schildersproject]{Voorbeeld: schildersproject}

\subsection{Taken}

\begin{frame}
	\begin{itemize}
		\item Deur schuren
		\item Hek schuren
		\item<2-> Muur verven
		\item<2-> Deur verven
		\item<2-> Hek verven
		\item<3-> Opruimen
	\end{itemize}
\end{frame}

\begin{frame}
	\begin{itemize}
		\item Deur schuren (3 uur)
		\item Hek schuren (2 uur)
		\item<2-> Muur verven (5 uur)
		\item<2-> Deur verven (2 uur)
		\item<2-> Hek verven (1 uur)
		\item<3-> Opruimen (1 uur)
	\end{itemize}
\end{frame}

\begin{frame}
	\begin{itemize}
		\item Taken zijn de onderdelen van het project
		\item Deze taken moet gepland worden
		\item Taken hebben een bepaalde duur
	\end{itemize}
\end{frame}

\subsection{Volgorde}
\begin{frame}
	\begin{itemize}
		\item Eerst schuren, daarna verven
		\item Eerst alle andere taken, daarna opruimen
		\item Volgorde is van belang
		\item In RCPSP heet dat: \emph{precedence constraints}
	\end{itemize}
\end{frame}

% demonstrate precedence constraints through graph
\begin{frame}
	\frametitle{Activity graph}
	\vspace{-1em}
	\begin{figure}[ht]
		\makebox[\textwidth][c]{\resizebox{.8\paperwidth}{!}{
			\inputtikz{activity_graph_simple}
		}}
		\vspace{-1em}
		\caption{Activity graph}
		\label{fig:activity_graph}
	\end{figure}
\end{frame}

\subsection{Schema}
\begin{frame}
	\begin{itemize}
		\item Plaatje invoegen
	\end{itemize}
\end{frame}

\section{Probleem}
\begin{frame}
	\begin{itemize}
		\item Schema werkt, wat gaat er dan toch fout?
		\item Middelen, of \emph{resources}
		\item De schilders hebben alleen:
		\item 1 schuurmachine
		\item 2 verfkwasten
	\end{itemize}
\end{frame}

\begin{frame}
	\begin{itemize}
		\item Schema werkt, wat gaat er dan toch fout?
		\item Middelen, of \emph{resources}
		\item De schilders hebben alleen:
		\item 1 schuurmachine
		\item 2 verfkwasten
	\end{itemize}
\end{frame}

% explain the terminology and concepts of RCPSP
\section{RCPSP}


% explain resource constraints
\subsection{Resource constraints}
\begin{frame}
	\frametitle{Resource constraints}
	\begin{itemize}
		\item Explain what a resource is
		\item Give painting brush example
		\item introduce resource set notation
		\item Explain what capacity is
		\item \mick{notation not necessary imo}
		\item Briefly mention the term usage (the concept of usage should already be clear)
		\item Make transition to the usage diagram, explain that there are more that the two mentioned activities in the project
	\end{itemize}
\end{frame}
	
% demonstrate how resources are linked to activities
\begin{frame}
	\frametitle{Resource usage}
	\vspace{-.2em}
	\begin{figure}[ht]
		\makebox[\textwidth][c]{\resizebox{.7\paperwidth}{!}{
			\inputtikz{usage}
		}}
		\vspace{-1em}
		\caption{The relation between activities and resources}
		\label{fig:activity_graph}
	\end{figure}
\end{frame}

% explain schedules
\subsection{Schedules}
\begin{frame}
	\frametitle{Schedules}
	\begin{itemize}
		\item What is a schedule
		\item What is feasibility
	\end{itemize}
\end{frame}

\begin{frame}
	\frametitle{Feasible schedule}
	\vspace{-1em}
	\begin{figure}[ht]
		\makebox[\textwidth][c]{\resizebox{.38\paperwidth}{!}{
			\inputtikz{schedule_feasible_profile}
		}}
		\vspace{-1.2em}
		\caption{The relation between activities and resources}
		\label{fig:activity_graph}
	\end{figure}
\end{frame}

\subsection{Summary}

\end{document}