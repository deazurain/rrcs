\documentclass{beamer}
\usepackage[latin1]{inputenc}
\usepackage[dutch]{babel}

\usepackage{amssymb}
\usepackage{amsmath}
\usepackage[numbers]{natbib}
\usepackage{xcolor}
% \usepackage[colorlinks,linkcolor=blue,urlcolor=blue,citecolor=blue]{hyperref}
\usepackage{mathrsfs}
\usepackage{comment}

\usepackage{tabularx}
\usepackage{booktabs}
\usepackage{amsthm}
\usepackage{nameref}
\usepackage{wrapfig}
\usepackage{pgf}
\usepackage{algorithm2e}

\theoremstyle{definition}
\providecommand{\SetAlgoLined}{\SetLine}
\providecommand{\DontPrintSemicolon}{\dontprintsemicolon}
\newcommand{\inputtikz}[1]{\input{tikz/#1}}
\inputtikz{preamble}
% long footnote section fixes
\dimen\footins=40\baselineskip\relax
\raggedbottom
\addtolength{\topskip}{0pt plus 10pt}
\interfootnotelinepenalty=10000
\newcommand{\TODO}[1]{{\color{red}\textbf{TODO: #1}}}
% \newtheorem{example}{Example}[section]
\newcommand{\res}[0]{\ensuremath{R}} %resources
\newcommand{\av}[2]{\ensuremath{av(r_{#1}, t_{#2})}} %availability of resource #1 at time #2
\newcommand{\capa}[1]{\ensuremath{cap(r_{#1})}} %capacity
\newcommand{\dur}[1]{\ensuremath{dur(v_{#1})}} %durability
\newcommand{\usage}[2]{\ensuremath{usage(v_{#1}, r_{#2})}} %usage of resource #2 by activity #1
\newcommand{\start}[1]{\ensuremath{start(v_{#1})}} %start time
\newcommand{\makespan}[1]{\ensuremath{C_{max}(#1)}} %makespan
\newcommand{\mindelay}[2]{\ensuremath{delay_{min}(t_{#1}, t_{#2})}} %minimum delay
\newcommand{\maxdelay}[2]{\ensuremath{delay_{max}(t_{#1}, t_{#2})}} %maximum delay
\newcommand{\weight}[2]{\ensuremath{weight(t_{#1}, t_{#2})}} %weight
% \newenvironment{definition}[1][Definition]{\begin{trivlist}
% \item[\hskip \labelsep {\bfseries #1}]}{\end{trivlist}}

% presentation theme
\usetheme{Szeged}

% lose the navigation buttons
\setbeamertemplate{navigation symbols}{}

% insert a local toc for each section
\AtBeginSection[] {
    \begin{frame}<beamer>
        \frametitle{Overzicht}
        \tableofcontents[currentsection,sectionstyle=show/hide,subsectionstyle=show/show/hide]
    \end{frame}
}

% number figures
\setbeamertemplate{caption}[numbered]

\title[RCPS]{Resource Constrained Project Scheduling\\
What is it and how can we approach it?}
\author{Mick van Gelderen}
\institute{TU Delft}
\date{\today}

\begin{document}

\begin{frame}
\titlepage
\end{frame}

\section*{Intro}
\begin{frame}
    \frametitle{Inleiding}
    \begin{itemize}
        \item Wie ben ik?
        \item Wat is het onderwerp van de paper?
        \item Met wie heb ik dat nog meer gedaan?
        \item Anekdote uit eigen leven: druk druk, veel plannen bla bla. Dat kan zonder geavanceerd algoritme omdat er de probleem instantie niet zo groot is. 
        \item Hoe plan je voor grote activiteiten? Voorbeeld trein onderhoud, voorbeeld luchthaven. 
        \item Geef aan dat er bijzondere voorwaarden zijn waar een schema aan moet voldoen.  Vandaar onderzoek naar een benaderings methode voor  RCPSP want het probleem is NP-hard. 
        \item Daar komen we allemaal op terug, nu eerst de precieze inhoud
    \end{itemize}
\end{frame}

\begin{frame}
    \frametitle{Inhoud}
    \tableofcontents
\end{frame}


\section{Activity graph}
\begin{frame}
	\frametitle{Test123}
	\vspace{-1em}
	\begin{figure}[ht]
		\makebox[\textwidth][c]{\resizebox{.7\paperwidth}{!}{
			\inputtikz{schedule_comparison}
		}}
		\vspace{-1em}
		\caption{Activity graph for the running example.}
		\label{fig:activity_graph}
	\end{figure}
\end{frame}


\end{document}