\documentclass{beamer}
\usepackage{etex}
\usepackage[latin1]{inputenc}
\usepackage[dutch]{babel}

\usepackage{amssymb}
\usepackage{amsmath}
\usepackage[numbers]{natbib}
\usepackage{xcolor}
% \usepackage[colorlinks,linkcolor=blue,urlcolor=blue,citecolor=blue]{hyperref}
\usepackage{mathrsfs}
\usepackage{comment}

\usepackage{tabularx}
\usepackage{booktabs}
\usepackage{amsthm}
\usepackage{nameref}
\usepackage{wrapfig}
\usepackage{pgf}
\usepackage{algorithm2e}
\usepackage{pgfpages}

\theoremstyle{definition}
\providecommand{\SetAlgoLined}{\SetLine}
\providecommand{\DontPrintSemicolon}{\dontprintsemicolon}
\newcommand{\inputtikz}[1]{\input{tikz/#1}}
\inputtikz{preamble}
% long footnote section fixes
\dimen\footins=40\baselineskip\relax
\raggedbottom
\addtolength{\topskip}{0pt plus 10pt}
\interfootnotelinepenalty=10000
\newcommand{\TODO}[1]{{\color{red}\textbf{TODO: #1}}}
\newcommand{\mick}[1]{{\color{red}\emph{#1}}}
% \newtheorem{example}{Example}[section]
\newcommand{\res}[0]{\ensuremath{R}} %resources
\newcommand{\av}[2]{\ensuremath{av(r_{#1}, t_{#2})}} %availability of resource #1 at time #2
\newcommand{\capa}[1]{\ensuremath{cap(r_{#1})}} %capacity
\newcommand{\dur}[1]{\ensuremath{dur(v_{#1})}} %durability
\newcommand{\usage}[2]{\ensuremath{usage(v_{#1}, r_{#2})}} %usage of resource #2 by activity #1
\newcommand{\start}[1]{\ensuremath{start(v_{#1})}} %start time
\newcommand{\makespan}[1]{\ensuremath{C_{max}(#1)}} %makespan
\newcommand{\mindelay}[2]{\ensuremath{delay_{min}(t_{#1}, t_{#2})}} %minimum delay
\newcommand{\maxdelay}[2]{\ensuremath{delay_{max}(t_{#1}, t_{#2})}} %maximum delay
\newcommand{\weight}[2]{\ensuremath{weight(t_{#1}, t_{#2})}} %weight
% \newenvironment{definition}[1][Definition]{\begin{trivlist}
% \item[\hskip \labelsep {\bfseries #1}]}{\end{trivlist}}

%\item<2-> laat dat item pas in deel 2 van een slide zien \item<3-6> tussen 3 en 6 etc
%\section[Kort]{Lange naam} zorgt ervoor dat in de bovenbalk de korte gebruikt wordt en op de slides en toc de lange

% presentation theme
\usetheme{Szeged}

% lose the navigation buttons
\setbeamertemplate{navigation symbols}{}

% insert a local toc for each section
%\AtBeginSection[] {
%    \begin{frame}<beamer>
%        \frametitle{\insertsection~overzicht}
%        \tableofcontents[currentsection,sectionstyle=hide/hide,subsectionstyle=show/show/hide]
%    \end{frame}
%}

% number figures
\setbeamertemplate{caption}[numbered]

% notes
\setbeameroption{show notes}
\setbeameroption{show notes on second screen=right}

\title[RCPSP]{An approach to Resource Constrained Project Scheduling}
\author{M. van Gelderen  \and
    R.M. de Lange \and
    B. Gris\`el \and
    F. van Tienen}
\institute{TU Delft}
\date{\today}

\begin{document}

\begin{frame}
\titlepage
\end{frame}

\section*{Intro}
\begin{frame}
  \frametitle{Inleiding}
  \begin{itemize}
		\item Veel schedulingsproblemen in dagelijks leven
		\item Kleine schaal
		\item Grote schaal
		\item Resource Constrained Project Scheduling Problem
  \end{itemize}
  \note{
Schemas van treinen\\
\begin{itemize}
	\item Treinen moeten worden onderhouden
	\item Gebeurt in een aantal werkplaatsen
	\item Werkplaats kan specialisatie hebben
	\item Idealiter vinden we een schema dat de beurten efficient pland
\end{itemize}
Schemas van vlieguitgen}
\end{frame}

\begin{frame}
    \frametitle{Inhoud}
    \tableofcontents
\end{frame}

% Scenario (intuitief, plaatje)
% Scenario (formeler, en wat willen we bereiken)
% Hoe gaan we dit oplossen?
% Om ons heen kijken => er zijn al mensen die een tijdsprobleem hebben opgelost in poly-tijd
% Laten we eens kijken of we dat kunnen gebruiken (hoe werkt zo'n STN)
% Ja want precedence constraints zijn om te schrijven naar STN constraints
% Er is een algoritme om dit op te lossen => oplossing laten zien
% Uit die oplossing een schema halen

\section{Scenario}

\begin{frame}
    \frametitle{Scenario}
    \begin{itemize}
    	\item We gaan een huis schilderen
    	\item De deur, het hek en buitenmuren moeten worden geschilderd
    	\item Hoe doen we dit zo effici\"{e}nt mogelijk?
    \end{itemize}
\end{frame}

\begin{frame}
    \frametitle{Taken}
    Wat er allemaal moet gebeuren:
     \begin{itemize}
    	\item Deur schilderen
	\item Hek schilderen
	\item Buitenmuren schilderen
	\item<2-> Deur schuren
	\item<2-> Hek schuren
	\item<3-> Materiaal schoonmaken en opruimen
    \end{itemize}
  \note{
Zijn de \textbf{activiteiten}.\\
Deur en hek van hout, geschuurd worden.\\
Alles schoonmaken.}
\end{frame}

\begin{frame}
    \frametitle{Taken}
    Wat er allemaal moet gebeuren:
     \begin{itemize}
    	\item Deur schilderen (2 uur)
	\item Hek schilderen (1 uur)
	\item Buitenmuren schilderen (5 uur)
	\item Deur schuren (3 uur)
	\item Hek schuren (2 uur)
	\item Materiaal schoonmaken (1 uur)
    \end{itemize}
\end{frame}

\begin{frame}
    	\frametitle{Taken}
   	Er zijn een aantal voorwaarden:
	\begin{enumerate}
	    	\item Houten oppervlak moet geschuurd worden voordat het kan worden geschilderd
		\item Materiaal schoonmaken moet als laatste
	\end{enumerate}
	\note{
Zijn de \textbf{precedence constraints}.}
\end{frame}

\subsection{Taken en Relaties}
\begin{frame}
	\frametitle{Taken en Relaties}
	\vspace{-1em}
	\begin{figure}[ht]
		\makebox[\textwidth][c]{\resizebox{.8\paperwidth}{!}{
			\inputtikz{precendence_graph}
		}}
		\vspace{-1em}
		\caption{Volgorde van de taken}
	\end{figure}
	\note{
Is een gereduceerde graaf.\\
Geen lijn tussen schuren en opruimen.\\
Worden ook precedence relaties genoemd.
}
\end{frame}

\subsection{Doel}
\begin{frame}
    	\frametitle{Doel}
   	\textbf{Gegeven}
	\begin{enumerate}
		\item Een set $V$ van taken $\{v_1,\dots,v_n\}$ met een bepaalde tijdsduur
		\item Een set $E$ van precedence relaties met elementen $(v_i, v_j)$ zodanig dat $v_i, v_j \in V$ en $v_i$ voor $v_j$ moet worden uitgevoerd
	\end{enumerate}
	
	\textbf{Gevraagd}\\
	Een schema waarin voor elke activiteit in $V$ een starttijd is toegewezen zodanig dat aan alle precedence relaties in $E$ voldaan wordt.
\end{frame}

\begin{frame}
	\begin{center}
		Is een soortgelijk probleem al opgelost?
	\end{center}
\end{frame}

\subsection{Simple Temporal Problem}
\begin{frame}
    	\frametitle{Simple Temporal Problem}
	
	\begin{itemize}
		\item Omstreeks 1990 gepubliceerd
		\item Is oplosbaar in polynomiale tijd
	\end{itemize}
   	
\end{frame}

\begin{frame}
    	\frametitle{Hoe werkt het}
	
	\begin{figure}[ht]
		\inputtikz{stn_precedence}
		\vspace{-1em}
		\caption{Volgorde vastgelegd in graaf}
	\end{figure}
	\note{Nogmaals de activiteiten en de precedence relaties. \\
		We gaan er een gewogen graaf van maken}
\end{frame}

\begin{frame}
    	\frametitle{Hoe werkt het}
	
	\begin{itemize}
		\item Aan elke kant wordt een minimale en maximale tijdsduur gehangen.
		\item Minimale en maximale tijdsduur naar elkaar toe brengen
		\item Doel: aan elke knoop een tijdstip toekennen
	\end{itemize}
	
	\note{
		Aan elke kant wordt een minimale en maximale tijdsduur gehangen zodanig dat het schema klopt als voor elke kant een waarde wordt gekozen die binnen de tijdsduur valt. \\
		Wat er wordt gedaan is de min en max tijdsduur beperken zodat dit geldt.\\
		Doel is dan om aan elke knoop een tijd toe te wijzen zdd de relaties in het schema kloppen.
	}
\end{frame}

\begin{frame}
    	\frametitle{Hoe werkt het}
	
	\begin{figure}[ht]
		\inputtikz{stn_weighted}
		\vspace{-1em}
		\caption{Gewichten toegevoegd}
	\end{figure}
	
	\note{
		Hier staat dat de tijd tussen elke activiteit minimaal 0 en maximaal INF moet zijn.\\
		Dit is eigenlijk een precedence relatie maar dan in tijd uitgedrukt.\\
		Hier wordt de tijdsduur van de activiteiten nog niet in meegenomen (NEXT).
	}
\end{frame}

\begin{frame}
    	\frametitle{Hoe werkt het}
	
	\begin{figure}[ht]
		\inputtikz{stn_expanded}
		\vspace{-1em}
		\caption{Extra knoop per taak}
	\end{figure}
	
	\note{
		Elke activiteit wordt weergegeven door twee knopen.\\
		1 voor het startpunt van de activiteit.\\
		1 voor het eindpunt van de activiteit.
	}
\end{frame}

\begin{frame}
    	\frametitle{Hoe werkt het}
	
	\begin{figure}[ht]
		\inputtikz{stn}
		\vspace{-1em}
		\caption{Gewichten voor duur van de taken}
	\end{figure}
	
	\note{
		Ook elke kant tussen het begin en eind van een taak krijgt een minimale en maximale limiet.\\
		Dit dwingt af dat de lengte van de activiteit precies hetzelfde is als in het originele model.
	}
\end{frame}

\begin{frame}
    	\frametitle{Hoe werkt het}
	
	\begin{figure}[ht]
		\makebox[\textwidth][c]{\resizebox{1\paperwidth}{!}{
			\inputtikz{stn_dummy}
		}}
		\vspace{-1em}
		\caption{Dummy start en eind toegevoegd}
	\end{figure}
	
	\note{
		Om ervoor te zorgen dat er 1 startpunt en 1 duidelijk eindpunt is van het schema, worden er twee dummy taken toegevoegd. 
	}
\end{frame}

\begin{frame}
    	\frametitle{De Oplossing}
	\begin{itemize}
		\item Essentie: tussen elk paar knopen het kortste pad berekenen
		\item Voor elke kant een minimale en maximale starttijd
		\item Vroegste starttijd algoritme
	\end{itemize}
	
	\note{
		Vroegste starttijd algoritme pakt voor elke knoop de vroegst mogelijke begintijd.
	}
\end{frame}

\begin{frame}
    	\frametitle{De Oplossing}
	\begin{figure}[ht]
		\inputtikz{schedule_infeasible}
		\vspace{-1em}
		\caption{Een schema dat aan de precedence constraints voldoet}
	\end{figure}
\end{frame}

\section{Probleem}
\subsection{Wat mist er nog?}
\begin{frame}
	\frametitle{Probleem}
	\begin{itemize}
		\item Volgorde van het schema klopt nu, wat mist er nog?
		\item<2->  De schilders hebben alleen:
		\item<2-> 1 schuurmachine
		\item<2-> 2 verfkwasten
		\item<3-> Middelen, of \emph{resources}
	\end{itemize}
	\note{
Er zijn niet genoeg \textbf{resources}.}
\end{frame}

\begin{frame}
	\frametitle{Resources}
	\begin{itemize}
		\item Resources zijn de gereedschappen voor het project
		\item Nodig om de taken uit te voeren
		\item Ze hebben een beschikbare capaciteit
		\item \emph{Resource} Constrained Project Scheduling Problem
	\end{itemize}
\end{frame}

\subsection{Resource Constrained Project Scheduling}
\begin{frame}
	\frametitle{Resource Constrained Project Scheduling}
	\textbf{Gegeven}
	\begin{enumerate}
		\item Een set $V$ van taken $\{v_1,\dots,v_n\}$ met een bepaalde tijdsduur en verbruik per resource
		\item Een set $E$ van precedence relaties met elementen $(v_i, v_j)$ zodanig dat $v_i, v_j \in V$ en $v_i$ voor $v_j$ moet worden uitgevoerd
		\item Een set $R$ van resources $\{r_1, \ldots, r_m\}$ met een bepaalde capaciteit
	\end{enumerate}
	
	\textbf{Gevraagd}\\
	Een schema waarin voor elke activiteit in $V$ een starttijd is toegewezen zodanig dat aan alle precedence relaties in $E$ voldaan wordt en een resource op geen enkel tijdstip over zijn capaciteit heen gaat.
	\note{
Taken hebben een \textbf{verbruik}.\\
\textbf{Capaciteit} van resources niet overschrijden.}
\end{frame}

% demonstrate how resources are linked to activities
\begin{frame}
	\frametitle{Resource gebruik}
	\vspace{-.2em}
	\begin{figure}[ht]
		\makebox[\textwidth][c]{\resizebox{.7\paperwidth}{!}{
			\inputtikz{usage_nl}
		}}
		\vspace{-1em}
		\caption{De relatie tussen taken en resources}
		\label{fig:activity_graph}
	\end{figure}
	\note{
Simpel probleem heeft ingewikkelde graaf.\\
Probleem is \textbf{moeilijk}!}
\end{frame}

\begin{frame}
\frametitle{Schema}
	\vspace{-1.2em}
	\begin{figure}[ht]
		\makebox[\textwidth][c]{\resizebox{.36\paperwidth}{!}{
			\inputtikz{schedule_infeasible_colored_profile}
		}}
		\vspace{-1.3em}
		\caption{Schema en resource gebruik}
		%\label{fig:activity_graph}
	\end{figure}
	\note{
De \textbf{deur} en het \textbf{hek} tegelijk schuren.}
\end{frame}

\subsection{Hoe kunnen we dit oplossen?}
\begin{frame}
	\frametitle{Hoe kunnen we dit oplossen?}
	\begin{itemize}
		\item Capaciteit van de resources in berekening
		\item Optimale schema in exponenti\"{e}le tijd
		\item RCPSP is een NP-hard probleem
	\end{itemize}
	\note{
We gaan naar een \textbf{voorbeeld}.}
\end{frame}

\begin{frame}
	\frametitle{Voorbeeld}
	\begin{itemize}
		\item 6-10 verschillende resources
		\item 30 treinen
		\item 20-30 taken
		\item<2-> 1 week plannen geeft \textbf{$600! \approx 1,27 * 10^{1408}$} mogelijkheden
		\item<3-> In praktijk is een snellere beslissing nodig
		\item<3-> Kunnen de resources ook worden weergegeven in de STN?
	\end{itemize}
	\note{
6-10 resource types, 30 treinen, 20-30 taken\\
Zou je constraints toe kunnen voegen aan je STN?\\
Manier om probleem te reduceren naar STN.
\textbf{$600!$}
}
\end{frame}

% explain the terminology and concepts of PCP
\section{Precedence Constraint Posting}

% explain resource constraints

\subsection{Hoe werkt PCP?}
\begin{frame}
	\frametitle{Precedence Constraint Posting}
	\begin{itemize}
		\item Piek detectie in de resources
		\item Conflict extractie
		\item Conflict oplossing
	\end{itemize}
	\note{
Een mogelijke manier is PCP.\\
Verschillende stappen:\\
\begin{itemize}
	\item Piek detectie in de resources
	\item Conflict extractie
	\item Conflict oplossing
\end{itemize}}
\end{frame}

\begin{frame}
	\frametitle{Piek detectie}
	\vspace{-1.2em}
	\begin{figure}[ht]
		\makebox[\textwidth][c]{\resizebox{.36\paperwidth}{!}{
			\inputtikz{schedule_infeasible_colored_profile}
		}}
		\vspace{-1.3em}
		\caption{Conflict in het schema}
		%\label{fig:activity_graph}
	\end{figure}
	\note{
Piek bij de \textbf{schuurmachine}.\\
Tussen \textbf{0 en 2}.}
\end{frame}

\begin{frame}
	\frametitle{Conflict extractie}
	\vspace{-1.2em}
	\begin{figure}[ht]
		\makebox[\textwidth][c]{\resizebox{.36\paperwidth}{!}{
			\inputtikz{schedule_infeasible_colored_profile_2}
		}}
		\vspace{-1.3em}
		\caption{Twee taken die voor conflict zorgen}
		%\label{fig:activity_graph}
	\end{figure}
	\note{
Twee activities, \textbf{deur en hek schuren}.
}
\end{frame}

\begin{frame}
	\frametitle{Conflict oplossing}
	\begin{itemize}
		\item Activiteiten $v_2$ en $v_1$ gebruiken beide resource $r_1$ in de piek
		\item Twee activiteiten die de schuurmachine gebruiken
		\item<2-> Volgorde: $v_1$ voor $v_2$
		\item<2-> Oftewel: eerst deur dan hek schuren
	\end{itemize}
	\note{
Precedence constraint toevoegen, $(v_1, v_2)$.}
\end{frame}

\subsection{Resultaat}
\begin{frame}
	\frametitle{Resultaat}
	\vspace{-0.2em}
	\begin{figure}[ht]
		\makebox[\textwidth][c]{\resizebox{.38\paperwidth}{!}{
			\inputtikz{schedule_feasible_profile}
		}}
		\vspace{-1.3em}
		\caption{Een uiteindelijk juist schema}
	\end{figure}
	\note{
Na oplossing via \textbf{kortste pad}, Floyd Warshall.\\
Zelfde manier als STN!}
\end{frame}

\subsection{Is dit de oplossing?}
\begin{frame}
	\frametitle{Is dit de oplossing?}
	\begin{itemize}
		\item Heuristisch algoritme
		\item Resource constraints naar tijd constraints
		\item Compleetheid en optimaliteit inruilen voor snelheid
		\item<2-> \textbf{There is no free lunch in computer science}
	\end{itemize}
	\note{
Er bestaat geen oplossing tenzij P=NP!\\
\textbf{There is no free lunch in computer science}}
\end{frame}
	

\section{Demo}
\begin{frame}
	\frametitle{Demo}
	\note{Opschieten met die demo!}
\end{frame}

\end{document}
