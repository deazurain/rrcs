\documentclass{beamer}
\usepackage{etex}
\usepackage[latin1]{inputenc}
\usepackage[dutch]{babel}

\usepackage{amssymb}
\usepackage{amsmath}
\usepackage[numbers]{natbib}
\usepackage{xcolor}
% \usepackage[colorlinks,linkcolor=blue,urlcolor=blue,citecolor=blue]{hyperref}
\usepackage{mathrsfs}
\usepackage{comment}

\usepackage{tabularx}
\usepackage{booktabs}
\usepackage{amsthm}
\usepackage{nameref}
\usepackage{wrapfig}
\usepackage{pgf}
\usepackage{algorithm2e}
\usepackage{pgfpages}

\theoremstyle{definition}
\providecommand{\SetAlgoLined}{\SetLine}
\providecommand{\DontPrintSemicolon}{\dontprintsemicolon}
\newcommand{\inputtikz}[1]{\input{tikz/#1}}
\inputtikz{preamble}
% long footnote section fixes
\dimen\footins=40\baselineskip\relax
\raggedbottom
\addtolength{\topskip}{0pt plus 10pt}
\interfootnotelinepenalty=10000
\newcommand{\TODO}[1]{{\color{red}\textbf{TODO: #1}}}
\newcommand{\mick}[1]{{\color{red}\emph{#1}}}
% \newtheorem{example}{Example}[section]
\newcommand{\res}[0]{\ensuremath{R}} %resources
\newcommand{\av}[2]{\ensuremath{av(r_{#1}, t_{#2})}} %availability of resource #1 at time #2
\newcommand{\capa}[1]{\ensuremath{cap(r_{#1})}} %capacity
\newcommand{\dur}[1]{\ensuremath{dur(v_{#1})}} %durability
\newcommand{\usage}[2]{\ensuremath{usage(v_{#1}, r_{#2})}} %usage of resource #2 by activity #1
\newcommand{\start}[1]{\ensuremath{start(v_{#1})}} %start time
\newcommand{\makespan}[1]{\ensuremath{C_{max}(#1)}} %makespan
\newcommand{\mindelay}[2]{\ensuremath{delay_{min}(t_{#1}, t_{#2})}} %minimum delay
\newcommand{\maxdelay}[2]{\ensuremath{delay_{max}(t_{#1}, t_{#2})}} %maximum delay
\newcommand{\weight}[2]{\ensuremath{weight(t_{#1}, t_{#2})}} %weight
% \newenvironment{definition}[1][Definition]{\begin{trivlist}
% \item[\hskip \labelsep {\bfseries #1}]}{\end{trivlist}}

%\item<2-> laat dat item pas in deel 2 van een slide zien \item<3-6> tussen 3 en 6 etc
%\section[Kort]{Lange naam} zorgt ervoor dat in de bovenbalk de korte gebruikt wordt en op de slides en toc de lange

% presentation theme
\usetheme{Szeged}

% lose the navigation buttons
\setbeamertemplate{navigation symbols}{}

% insert a local toc for each section
\AtBeginSection[] {
    \begin{frame}<beamer>
        \frametitle{\insertsection~overzicht}

        \tableofcontents[currentsection,sectionstyle=hide/hide,subsectionstyle=show/show/hide]
    \end{frame}
}

% number figures
\setbeamertemplate{caption}[numbered]

% notes
\setbeameroption{show notes}
\setbeameroption{show notes on second screen=right}

\title[RCPS]{Resource Constrained Project Scheduling\\
An Approach}
\author{M. van Gelderen  \and
    R.M. de Lange \and
    B. Gris\`el \and
    F. van Tienen}
\institute{TU Delft}
\date{\today}

\begin{document}

\begin{frame}
\titlepage
\end{frame}

\section*{Intro}
\begin{frame}
  \frametitle{Inleiding}
  \begin{itemize}
		\item Veel planningsproblemen in dagelijks leven
		\note{bla }
		\item Kleine schaal
		\item Grote schaal
		\item Resource Constrained Project Scheduling Problem
  \end{itemize}
\end{frame}

\begin{frame}
    \frametitle{Inhoud}
    \tableofcontents
\end{frame}

% Scenario (intuitief, plaatje)
% Scenario (formeler, en wat willen we bereiken)
% Hoe gaan we dit oplossen?
% Om ons heen kijken => er zijn al mensen die een tijdsprobleem hebben opgelost in poly-tijd
% Laten we eens kijken of we dat kunnen gebruiken (hoe werkt zo'n STN)
% Ja want precedence constraints zijn om te schrijven naar STN constraints
% Er is een algoritme om dit op te lossen => oplossing laten zien
% Uit die oplossing een schema halen

\section{Scenario}

\begin{frame}
    \frametitle{Scenario}
    PLAATJE VAN HUIS MET HEK EN DEUR HIER
    
    \begin{center}
    	De deur, het hek en buitenmuren moeten worden geschilderd
    \end{center}
\end{frame}

\begin{frame}
    \begin{center}
    	Hoe doen we dit zo effici\"{e}nt mogelijk?
    \end{center}
\end{frame}

\subsection{Taken}
\begin{frame}
    \frametitle{Taken}
    Wat er allemaal moet gebeuren:
     \begin{itemize}
    	\item Deur schilderen
	\item Hek schilderen
	\item Buitenmuren schilderen
	\item<2-> Deur schuren
	\item<2-> Hek schuren
	\item<3-> Materiaal schoonmaken
    \end{itemize}
\end{frame}

\begin{frame}
    \frametitle{Taken}
    Wat er allemaal moet gebeuren:
     \begin{itemize}
    	\item Deur schilderen (2 uur)
	\item Hek schilderen (1 uur)
	\item Buitenmuren schilderen (5 uur)
	\item Deur schuren (3 uur)
	\item Hek schuren (2 uur)
	\item Materiaal schoonmaken (1 uur)
    \end{itemize}
\end{frame}

\begin{frame}
    	\frametitle{Taken}
   	Er zijn een aantal voorwaarden:
	\begin{enumerate}
	    	\item Een oppervlak moet eerst geschuurd worden voordat het kan worden geschilderd
		\item Materiaal schoonmaken doe je als laatste
	\end{enumerate}
\end{frame}

\subsection{Taken en Relaties}
\begin{frame}
	\frametitle{Taken en Relaties}
	\vspace{-1em}
	\begin{figure}[ht]
		\makebox[\textwidth][c]{\resizebox{.8\paperwidth}{!}{
			\inputtikz{precendence_graph}
		}}
		\vspace{-1em}
	\end{figure}
\end{frame}

\subsection{Doel}
\begin{frame}
    	\frametitle{Doel}
   	\textbf{Gegeven}
	\begin{enumerate}
		\item Een set $V$ van taken $\{v_1,\dots,v_n\}$ met een bepaalde tijdsduur
		\item Een set $E$ van precedence constraints met elementen $(v_i, v_j)$ zodanig dat $v_i, v_j \in V$ en $v_i$ voor $v_j$ moet worden uitgevoerd
	\end{enumerate}
	
	\textbf{Gevraagd}\\
	Een schema waarin voor elke activiteit in $V$ een starttijd is toegewezen zodanig dat aan alle precendence relaties in $E$ voldaan wordt.
\end{frame}

\begin{frame}
	\begin{center}
		Is er een probleem wat hier op lijkt? Heeft iemand een soortgelijk probleem al opgelost?
	\end{center}
\end{frame}

\begin{frame}
    	\frametitle{Simple Temporal Problem}
   	
\end{frame}

\section{Probleem}
\subsection{Wat klopt er nog niet?}
\begin{frame}
\frametitle{Probleem}
	\begin{itemize}
		\item Volgorde van het schema klopt, wat gaat er dan toch fout?
		\item<2->  De schilders hebben alleen:
		\item<2-> 1 schuurmachine
		\item<2-> 2 verfkwasten
		\item<3-> Middelen, of \emph{resources}
	\end{itemize}
\end{frame}

\begin{frame}
\frametitle{Schema}
	\vspace{-1.2em}
	\begin{figure}[ht]
		\makebox[\textwidth][c]{\resizebox{.36\paperwidth}{!}{
			\inputtikz{schedule_infeasible_colored_profile}
		}}
		\vspace{-1.3em}
		\caption{Schema zonder en met resource constraints}
		%\label{fig:activity_graph}
	\end{figure}
\end{frame}

\begin{frame}
	\begin{itemize}
		\item Middelen zijn de gereedschappen voor het project
		\item Nodig om de taken uit te voeren
		\item Ze hebben een beschikbare capaciteit
	\end{itemize}
\end{frame}

\begin{frame}
	\frametitle{Probleem}
	\textbf{Gegeven}
	\begin{enumerate}
		\item Een set $V$ van taken $\{v_1,\dots,v_n\}$ met een bepaalde tijdsduur en verbruik per resource
		\item Een set $E$ van precedence constraints met elementen $(v_i, v_j)$ zodanig dat $v_i, v_j \in V$ en $v_i$ voor $v_j$ moet worden uitgevoerd
		\item Een set $R$ van resources $\{r_1, \ldots, r_m\}$ met een bepaalde capaciteit
	\end{enumerate}
	
	\textbf{Gevraagd}\\
	Een schema waarin voor elke activiteit in $V$ een starttijd is toegewezen zodanig dat aan alle precendence relaties in $E$ voldaan wordt en een resource op geen enkel tijdstip over zijn capaciteit heen gaat.
\end{frame}

\begin{frame}
	\begin{itemize}
		\item \emph{Resource} Constrained Project Scheduling Problem
		\item Capaciteit van de middelen meenemen in berekening
		\item Berekenen van het optimale schema gaat in exponenti\"{e}le tijd
		\item RCPSP is een NP-hard probleem
	\end{itemize}
\end{frame}

\subsection{Hoe kunnen we dit oplossen?}
\begin{frame}
	\begin{itemize}
		\item Dit is de essentie van RCPSP
		\item In praktijk is een snellere beslissing nodig
		\item Kunnen de middelen ook worden weergegeven in de STN?
	\end{itemize}
\end{frame}

% explain the terminology and concepts of PCP
\section{PCP}

% explain resource constraints

\subsection{PCP}
\begin{frame}
	\frametitle{PCP}
	\textbf{Precedence Constraint Posting}
	\begin{itemize}
		\item<2-> Piek detectie in de resources
		\item<3-> Conflict extractie
		\item<4-> Conflict oplossing
	\end{itemize}
\end{frame}

\begin{frame}
	\frametitle{Piek detectie}
	\vspace{-1.2em}
	\begin{figure}[ht]
		\makebox[\textwidth][c]{\resizebox{.36\paperwidth}{!}{
			\inputtikz{schedule_infeasible_colored_profile}
		}}
		\vspace{-1.3em}
		\caption{Schema zonder en met resource constraints}
		%\label{fig:activity_graph}
	\end{figure}
\end{frame}

\begin{frame}
	\frametitle{Conflict extractie}
	\vspace{-1.2em}
	\begin{figure}[ht]
		\makebox[\textwidth][c]{\resizebox{.36\paperwidth}{!}{
			\inputtikz{schedule_infeasible_colored_profile}
		}}
		\vspace{-1.3em}
		\caption{Schema zonder en met resource constraints}
		%\label{fig:activity_graph}
	\end{figure}
\end{frame}

\begin{frame}
	\frametitle{Conflict oplossing}
	\begin{itemize}
		\item Activiteiten $v_2$ en $v_1$ gebruiken beide resource $r_1$ in de piek
		\item Constraint tussen $v_1$ en $v_2$
	\end{itemize}
\end{frame}

\begin{frame}
	\frametitle{Uiteindelijke schema}
	\vspace{-0.2em}
	\begin{figure}[ht]
		\makebox[\textwidth][c]{\resizebox{.38\paperwidth}{!}{
			\inputtikz{schedule_feasible_profile}
		}}
		\label{fig:activity_graph}
	\end{figure}
\end{frame}
	
% demonstrate how resources are linked to activities
\begin{frame}
	\frametitle{Resource usage}
	\vspace{-.2em}
	\begin{figure}[ht]
		\makebox[\textwidth][c]{\resizebox{.7\paperwidth}{!}{
			\inputtikz{usage}
		}}
		\vspace{-1em}
		\caption{The relation between activities and resources}
		\label{fig:activity_graph}
	\end{figure}
\end{frame}

\subsection{Summary}

\end{document}