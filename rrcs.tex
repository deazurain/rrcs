\documentclass{article}
\usepackage{a4wide}
\usepackage{amssymb}

\newcommand{\leading}[1]{\textbf{#1}}
\usepackage{xcolor}

%% if your are not using LaTeX2e use instead
%% \documentstyle[bnaic]{article}

%% begin document with title, author and affiliations

\title{Robust resource constraint scheduling}
\author{M. van Gelderen  \and
    R.M. de Lange \and
    B. Gris\`el \and
    F. van Tienen}

\pagestyle{empty}

\newcommand{\TODO}[1]{{\color{red}\textbf{TODO: #1}}}

\begin{document}
\maketitle
\thispagestyle{empty}

\begin{abstract}
This is the abstract of my paper.
This is the abstract of my paper.
This is the abstract of my paper.
This is the abstract of my paper.
This is the abstract of my paper.
This is the abstract of my paper.
This is the abstract of my paper.
This is the abstract of my paper.
This is the abstract of my paper.
This is the abstract of my paper.
This is the abstract of my paper.
This is the abstract of my paper.
\end{abstract}


\section{Introduction}

%% References needed

The resource scheduling problem is a problem concerned with the scheduling of tasks on a specific set of resources. These, often scarce, resources form a constraint on the volume of tasks which can be processed at any one time. Scheduling focuses on optimizing a schedule to either consume the least amount of resources, or perform all tasks in minimum required time. There are several different forms of resource constrained scheduling, all dealing with a specific type of scheduling problem.

The main aspects that will be covered in this paper are the precedence constraint and temporal constraint. Precedence constraints require a set of tasks to be performed in a specific order, requiring one task to be performed before the other. Temporal constraints pose a constraint on start and finishing time of one or several tasks, requiring tasks to be performed within a specific time interval. \TODO{Ik dacht dat temporal constraints een generalisatie van precedence constraints waren; ze bepalen ook de volgorde van activities maar dan met een min en max tussentijd} Continuous planning implies another difficulty in the scheduling problem and will also be a subject of this paper. In this case the set of tasks to be performed is not closed, meaning that during execution of the schedule the set of tasks may change and with it the optimal solution. However, it may also be undesirable to introduce major changes in the baseline schedule.

The matter of robust resource constrained scheduling requires that the schedule not only gives an efficient solution to perform all tasks. It  should also take in account that some jobs might be delayed or resources become unavailable due to circumstances and then still give an efficient solution to the problem. These kinds of changes can also occur due to the continuous nature of the application in which the scheduling is applied, for example by adding new tasks to the set.

In the following chapter we will be dealing with the theory on these matters and discus several research papers that cover the topic. There has been a substantial amount of research performed on the subject of resource constrained scheduling and also many papers have been published discussing the matter. In this paper a selection of these researches will be covered to give a good basis on the problems that can be found in practice.

We will take a deeper look at the practical applications for the theory, with their respective overlap and conflicts with the aforementioned theory. As one might expect, several issues occur in day-to-day applications of scheduling and not every aspect that is found in practice is covered by theoretical research on this subject. As is not unusual for mathematical or logical models of real-life problems, there is aways some assumption on preconditions which needs to be done to make a simplified model.

\newpage

\section{Problem background}

Different aspects of the underlying theory leading to RRCS.  Per point explanation of the theory with references to relevant papers. This will be the main section of the paper.

\TODO{Describe what notation conventions we will adopt. I suggest using the notation from P. Brucker et al, Eauropean Journal of Operational Research 112 (1999). }

\subsection{Prerequisites}
In this section, the general problem focussed on in this paper is explained. Projects composed of activities, resources and constraints are introduced and a unifying notation is presented. First, some terminology is discussed followed by a problem definition.

\subsubsection{Activities}
Each project consists of a finite set of activities (also called jobs). Each activity has a processing time $p_{j} \in \mathbb{N}$; the time needed to complete the activity. The set of all activities is $V$.

\subsubsection{Resources}
An activity requires a certain amount of resources to be available upon execution. For example, changing a tyre might require two workman and one car lift. In general, there are two kinds of resources: renewable and nonrenewable resources. With a nonrenewable resources (such as petrol), once a part of a resource is used, it will no longer be available. This contrasts with renewable resources, which do not deplete when used. For renewable resources, the amount of units available of resource $k$ at any given time is $\mathcal{R}^{\rho}_k$ (for example the number of machines in a shop). The total amount of units available for a nonrenewable resource $k$ is noted by $\mathcal{R}^{v}_k$. The set of all renewable resources within a project is indicated with $\mathcal{R}^{\rho}$ and the set of renewable resources is indicated by $\mathcal{R}^{v}$. 

Activities consume a certain amount of one or more resources. The amount that an activity $j$ consumes of a resource $k$ during a unit of time is denoted by $r^{\rho}_{jk}$ for a renewable resource and  $r^{v}_{jk}$ for an nonrenewable resource. 

\subsubsection{Contraints}
Apart from activities that consume resources, additional constraints can be added. Two of these are precedence constraints and temporal constrains. Some activities need to be executed before another activity can be executed (for example, a surface needs to be sanded before it can be painted). Such activity precedence can be visualized using a so-called precedence graph $(V, E)$ in which every activity is represented by a node and directed edges between nodes $A_i$ and $A_j$ are added if an action $A_i$ needs to be executed before $A_j$.

Temporal constraints can be used to specify a minimum and maximum time lag between two activities $i$ and $j$. The notation used in this paper is $d^{min}_{ij}$ and $d^{max}_{ij}$ respectively.

\subsubsection{Schedule}
The general idea of resource scheduling is to minimize the project execution time without over-utilising the resources or violate project constraints like the constraints discussed above. The project itself can also have a deadline, the total amount of time units before the project needs to be completed. A mapping of a unit of time in the model and an actual timespan in the real world can be indicated by the Time Horizon $t=1,2,3,. . .,T$ and corresponding time intervals $[t-1,t]$. 

\subsection{STN}
\TODO{WTF is dit: Bastiaan}

\subsection{Formal Definition}
\TODO{Freek, naar voren halen}
The general definition for the robust resource constraint scheduling problem is defined by a tuple $(V, p, E,R,B, b)$:\\

Given:
\begin{itemize}
\item a set of activities $V$
\item the durations of the activities $(p_1\ldots p_n)$
\item a set of constraints which needs to be satisfied $E$
\item a set of available resources $\mathcal{R}$
\item the availabilities of the resources $B$
\item the demands for an activity $j$ of resource $k$ $(r_{11}\ldots r_{jk})$
\end{itemize}

\TODO{Explain what the question of the resource constraint scheduling is}

\subsection{Problem Complexity}
\TODO{Freek}
\subsubsection{Problem Classification}
RRCS belongs to the class of problems that are $NP-hard$. According to the complexity theory a problem belongs to the class $NP-hard$ if its decision version is $NP-complete$. We will use this theory to prove that RRCS belongs to the class $NP-hard$.

\TODO{Define the decision variant of the RRCS problem and show through reduction that the decision variant is NP-complete.}
\subsubsection{Time Complexity}
\subsubsection{Space Complexity}

\section{Constraint posting}
\TODO{Bastiaan, verdelen in subkpjes}

\section{Continuous planning}
\TODO{Mick de LangeR}

\section{Contraint Posting in Continuous planning}
\TODO{Mick van Gelderen}

\subsection{Conclusion}
\TODO{Mick de Lange-R}

\section{Problem Practice}
This section will discuss how the practice problems can be dealt with using the theory described in the previous section. First, the problems that can be solved using this theory are discussed. After that, the problems which cannot be solved directly using the theory are explained.\\

\begin{itemize}
\item Project scheduling under uncertainty: Survey
and research potentials
\item A Comparison of Heuristic and Optimum Solutions
\item Reactive scheduling, improving the robustness of schedules
\end{itemize}

\subsection{Solvable problems with PCP}
\TODO{Freek}
\begin{itemize}
\item What are the examples from the real world? 
\item What are the underlying problems of these examples?
\item How can they be solved using the theory described above?
\end{itemize}

\subsection{Related problems with PCP}
\TODO{Bastiaan}
\begin{itemize}
\item What are the examples from the real world?
\item Trains, trucks, maintenance scheduling, etc.
\item Airport planning / different factors: maintenance / tanking / bagage / boarding
\item What are the underlying problems of these examples?
\item How does the above theory needs to be augmented in order to be able to solve the problems?
\end{itemize}

\section{Conclusion}
The overall conclusion.

\section{\TODO{Remove} Making References}

  Make references in the running text with the \verb+\cite+
  command \cite{dijkstra68}. Multiple referrences go like this
  \cite{charniak85,steels98}.


\bibliographystyle{plain}
\bibliography{references}

\end{document}








