\documentclass{article}
\usepackage{a4wide}


%% if your are not using LaTeX2e use instead
%% \documentstyle[bnaic]{article}

%% begin document with title, author and affiliations

\title{Seminarium TI3700\\Robust resource constraint scheduling}
\author{Mick van Gelderen  \and
    Mick de Lange \and
    Bastiaan Grisel \and
    Freek van Tienen}
\date{}

\pagestyle{empty}

\begin{document}
\maketitle
\thispagestyle{empty}

\begin{abstract}
This is the abstract of my paper.
This is the abstract of my paper.
This is the abstract of my paper.
This is the abstract of my paper.
This is the abstract of my paper.
This is the abstract of my paper.
This is the abstract of my paper.
This is the abstract of my paper.
This is the abstract of my paper.
This is the abstract of my paper.
This is the abstract of my paper.
This is the abstract of my paper.
\end{abstract}


\section{Introduction}

An introduction about what we are going to explain in the paper.

\begin{itemize}
\item RCSP
\item Precedence Constraint Scheduling
\item Continuplanning
\end{itemize}


\section{Problem theory}

Different aspects of the underlying theory leading to RRCS.  Per point explanation of the theory with references to relevant papers. This will be the main section of the paper.

\subsection{Research}

\begin{itemize}
\item Lombardi
\item Policella
\item Some other papers about continuplanning
\end{itemize}

\subsection{Conclusion}

Short recap of previous section. Briefly goes through the steps listed above.

\section{Problem Practice}
This section will discuss how the practice problems can be dealt with using the theory described in the previous section. First, the problems that can be solved using this theory are discussed. After that, the problems which cannot be solved directly using the theory are explained.\\

\begin{itemize}
\item Project scheduling under uncertainty: Survey
and research potentials
\item A Comparison of Heuristic and Optimum Solutions
\item Reactive scheduling, improving the robustness of schedules
\end{itemize}

\subsection{Solvable problems}
\begin{itemize}
\item What are the examples from the real world?
\item What are the underlying problems of these examples?
\item How can they be solved using the theory described above?
\end{itemize}

\subsection{Not yet solvable problems}
\begin{itemize}
\item What are the examples from the real world?
\item Trains, trucks, maintenance scheduling, etc.
\item Airport planning / different factors: maintenance / tanking / bagage / boarding
\item What are the underlying problems of these examples?
\item How does the above theory needs to be augmented in order to be able to solve the problems?
\end{itemize}

\section{Conclusion}

The overall conclusion.

\section{TOBE REMOVED: Making References}

  Make references in the running text with the \verb+\cite+
  command \cite{dijkstra68}. Multiple referrences go like this
  \cite{charniak85,steels98}.


\bibliographystyle{plain}
\bibliography{references}

\end{document}








