\documentclass{article}
\usepackage{a4wide}
\usepackage{xcolor}

%% if your are not using LaTeX2e use instead
%% \documentstyle[bnaic]{article}

%% begin document with title, author and affiliations

\title{Robust resource constraint scheduling}
\author{M. van Gelderen  \and
    R.M. de Lange \and
    B. Gris\`el \and
    F. van Tienen}

\pagestyle{empty}

\newcommand{\TODO}[1]{{\color{red}\textbf{TODO: #1}}}

\begin{document}
\maketitle
\thispagestyle{empty}

\begin{abstract}
This is the abstract of my paper.
This is the abstract of my paper.
This is the abstract of my paper.
This is the abstract of my paper.
This is the abstract of my paper.
This is the abstract of my paper.
This is the abstract of my paper.
This is the abstract of my paper.
This is the abstract of my paper.
This is the abstract of my paper.
This is the abstract of my paper.
This is the abstract of my paper.
\end{abstract}


\section{Introduction}

%% References needed

The resource scheduling problem is a problem concerned with the scheduling of tasks on a specific set of resources. These, often scarce, resources form a constraint on the volume of tasks which can be processed at any one time. Scheduling focuses on optimizing a schedule to either consume the least amount of resources, or perform all tasks in minimum required time. \TODO{1. Zijn dit alle mogelijkheden? 2. Op welke vormen gaan we in, is het nuttig om andere te noemen?} There are several different forms of resource constrained scheduling, all dealing with a specific type of scheduling problem.

The main aspects that will be covered in this paper are the precedence constraint and temporal constraint. Precedence constraints require a set of tasks to be performed in a specific order, requiring one task to be performed before the other. Temporal constraints pose a constraint on start and finishing time of one or several tasks, requiring tasks to be performed within a specific time interval. \TODO{Ik dacht dat temporal constraints een generalisatie van precedence constraints waren; ze bepalen ook de volgorde van activities maar dan met een min en max tussentijd} Continuous planning implies another difficulty in the scheduling problem and will also be a subject of this paper. In this case the set of tasks to be performed is not closed, meaning that during execution of the schedule the set of tasks may change and with it the optimal solution. \TODO{It is often not desirable however to introduce a lot of changes with respect to the baseline schedule?}

The matter of robust resource constrained scheduling requires that the schedule not only gives an efficient solution to perform all tasks. It  should also take in account that some jobs might be delayed or resources become unavailable \TODO{Michel gaf aan dat er geen rekening gehouden wordt met het onbeschikbaar worden van resources in zijn mail dacht ik. Misschien is het toch wél handig inderdaad om dit te noemen omdat het mij geen onpractisch aspect lijkt om rekening mee te houden} due to circumstances and then still give an efficient solution to the problem. These kinds of changes can also occur due to the continuous nature of the application in which the scheduling is applied, for example by adding new tasks to the set.

In the following chapter we will be dealing with the theory on these matters and discus several research papers that cover the topic. There has been a substantial amount of research performed on the subject of resource constrained scheduling and also many papers have been published discussing the matter. In this paper a selection of these researches will be covered to give a good basis on the problems we will come by in practice. \TODO{Informeel we will come by}

We will take a deeper look at the practical applications for the theory, with their respective overlap and conflicts with the aforementioned theory. As one might expect, several issues occur in day-to-day applications of scheduling and not every aspect that is found in practice is covered by theoretical research on this subject. As is not unusual for mathematical or logical models of real-life problems, there is aways some assumption on preconditions which needs to be done to make a simplified model.


\newpage

\section{Problem theory}

Different aspects of the underlying theory leading to RRCS.  Per point explanation of the theory with references to relevant papers. This will be the main section of the paper.

\TODO{Describe what notation conventions we will adopt. I suggest using the notation from P. Brucker et al, Eauropean Journal of Operational Research 112 (1999). }

\subsection{Prerequisites}
\subsubsection{Activities}
Activities and processing, start/completion times. 
\subsubsection{Resources}
Explanation of renewable and non-renewable resources.
\subsubsection{Contraints}
\subsubsection{Resource Constraints}
\subsubsection{Precedence Constraints}
\subsubsection{Temporal Constraints}
\subsubsection{Schedule}
time-feasable/resource-feasable schedules. Deadlines, time horizon.

\subsection{Formal Definition}
Actual definition of the problem and formal notation

\subsection{Problem Complexity}
\subsubsection{Problem Classification}
Explain NP-hardness of the problem
\subsubsection{Time Complexity}
\subsubsection{Space Complexity}

\subsection{Optimal Algorithm}
Discuss the optimal algorithm and how it cannot be used and is insensitive to technological speedups.

\subsection{Problem Extension}
The basic PS problem will sometimes not suffice. Several extensions to the PS problem have been made of which some will be discussed here. 
\subsubsection{Multi-mode project scheduling}
\TODO{Reference Herroelen, Reactive scheduling in the multi-mode RCPSP}

\subsubsection{Stochastic activity durations}
\subsection{Approximation Algorithms}


\subsection{Research}

\begin{itemize}
\item Lombardi
\item Policella
\item Some other papers about continuous planning
\end{itemize}

\subsection{Conclusion}

Short recap of previous section. Briefly goes through the steps listed above.

\section{Problem Practice}
This section will discuss how the practice problems can be dealt with using the theory described in the previous section. First, the problems that can be solved using this theory are discussed. After that, the problems which cannot be solved directly using the theory are explained.\\

\begin{itemize}
\item Project scheduling under uncertainty: Survey
and research potentials
\item A Comparison of Heuristic and Optimum Solutions
\item Reactive scheduling, improving the robustness of schedules
\end{itemize}

\subsection{Solvable problems}
\begin{itemize}
\item What are the examples from the real world?
\item What are the underlying problems of these examples?
\item How can they be solved using the theory described above?
\end{itemize}

\subsection{Not yet solvable problems}
\begin{itemize}
\item What are the examples from the real world?
\item Trains, trucks, maintenance scheduling, etc.
\item Airport planning / different factors: maintenance / tanking / bagage / boarding
\item What are the underlying problems of these examples?
\item How does the above theory needs to be augmented in order to be able to solve the problems?
\end{itemize}

\section{Conclusion}

The overall conclusion.


\section{\TODO{Remove} Making References}

  Make references in the running text with the \verb+\cite+
  command \cite{dijkstra68}. Multiple referrences go like this
  \cite{charniak85,steels98}.


\bibliographystyle{plain}
\bibliography{references}

\end{document}








