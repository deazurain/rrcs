\documentclass{article}
\usepackage{a4wide}
\usepackage{amssymb}
\usepackage{amsmath}
\usepackage[numbers]{natbib}
\usepackage{xcolor}
\usepackage{hyperref}
\usepackage{mathrsfs}

\title{Precedence Constraint Posting in Continuous Planning; Practical Adoption of the Scheduling Problem.}
\author{M. van Gelderen  \and
    R.M. de Lange \and
    B. Gris\`el \and
    F. van Tienen}
\date{}

\pagestyle{empty}

\newcommand{\TODO}[1]{{\color{red}\textbf{TODO: #1}}}

\newcommand{\renres}[0]{\mathscr{R}^\rho} %renewable resource
\newcommand{\conres}[0]{\mathscr{R}^\nu} %consumable resource
% heb er een aantal uitgeprobeerd, deze komt het dichts bij al is ie te slanted
% naar mijn smaak. 
%$\mathbb{R}^\nu$
%$\mathbb{R}^v$
%$\mathscr{R}^\upsilon$
%$\mathcal{R}^\nu$

%\setlength{\parindent}{0pt}
%\setlength{\parskip}{\baselineskip}

\begin{document}
\maketitle
\thispagestyle{empty}

\begin{abstract}
\TODO{abstract als laatste schrijven}
\end{abstract}

\newpage

\section{Scope and purpose}
\TODO{rewrite introduction after draft finished, explain structure, chapters etc.}\\

\TODO{Ik vond "Scope and purpose" wel een mooie alternatieve titel voor dit stuk}
In het kort begrijp ik het volgende van wat we doen: 

We bekijken precedence constraint posting als methode voor het genereren van een schema voor het continuous resource constrained project scheduling probleem. Het doel van de paper is om een volledige informatie bron zijn voor een 3e jaars TI student over het hierboven genoemde

Eerst geven we een intuïtieve definitie van het RCPSP. Dan volgt een formele definitie van RCPSP. Hierna volgt een intuïtieve definitie van STNs, daarna volgt een formele definitie van STNs. 

Dan kunnen we de diepte in en leggen we uit wat precedence constraint posting is. Dan maken we de transitie naar continuous planning en waarom het nodig is. Hierna leggen we uit waarom PCP geschikt is om continuous planning mee te realiseren. \\



The resource scheduling problem is a problem concerned with the scheduling of tasks on a specific set of resources.
These, often scarce, resources form a constraint on the volume of tasks which can be processed at any one time.
Scheduling focuses on optimizing a schedule to either consume the least amount of resources, or perform all tasks in minimum required time.  \cite{brucker99}
There are several different forms of resource constrained scheduling, all dealing with a specific type of scheduling problem.

The main aspects that will be covered in this paper are precedence constraint posting and continuous planning.
Precedence constraints require a set of tasks to be performed in a specific order, requiring one task to be performed before the other.
Also, temporal constraints have a part in this scheduling problem, but this will not be the focus in this paper.
Continuous planning is a form of scheduling where the scheduling process is not completed, the process continuous while schedule is being executed.
In this case the set of tasks to be performed is not closed, meaning that during execution of the schedule the set of tasks may change and with it the optimal solution.
However, it may also be undesirable to introduce major changes in the baseline schedule.
Therefore, the continuous planning aspect of scheduling is one of the more complex issues and also a quite common phenomenon in practical situations.

In this paper we will combine these to elements in the scheduling problem and study this in the light of robust resource constrained scheduling.
Robust resource constrained scheduling requires that the schedule not only gives an efficient solution to perform all tasks, it should also take in to account external changes.
In this paper we will use the same definition of robust as in \cite{policella07} : ``{(1) the ability to absorb exogenous and/or unforeseen events without loss of consistency, and (2) the ability to keep the pace with the execution guaranteeing a prompt answer to the various events.}''
These kinds of changes can also occur due to the continuous nature of the application in which the scheduling is applied, for example by adding new tasks to the set.

In the following chapter we will be dealing with the theory on scheduling problems and discus several relevant research papers that cover the topic.
After which the two main aspects, precedence constraint posting and continuous planning, will be discussed further and then combined and linked to practical problems.
We will take a deeper look at the practical applications for the theory, with their respective overlap and conflicts with the aforementioned theory.
As one might expect, several issues occur in day-to-day applications of scheduling and not every aspect that is found in practice is covered by theoretical research on this subject.
As is not unusual for mathematical or logical models of real-life problems, there is aways some assumption on preconditions which needs to be done to make a simplified model.

This paper is intended to link the practical applications of the scheduling problem to several aspects of the theory, thereby addressing the incompleteness of the problem theory and discover what aspects need further research.

\newpage

\section{Problem background}
\TODO{Freek}

\TODO{MickL}

\TODO{rewrite, more text for readers management, complete definition \& make formal, use normal R, reference, explain}

\TODO{define 'optimal'}

Different aspects of the underlying theory leading to RRCS.
 Per point explanation of the theory with references to relevant papers.
This will be the main section of the paper.

\TODO{Describe what notation conventions we will adopt.
I suggest using the notation from P. Brucker et al, European Journal of Operational Research 112 (1999). }

Sched
\subsection{Formal Definition}
The general definition for the resource constraint scheduling problem is defined by a tuple $(T, V, W, E, R)$: \cite{brucker99}

Given:
\begin{itemize}
\item a horizon $T$ that denotes the total time horizon for the problem, where $T \in \mathbb{N}$
\item a set of activities(jobs) $V = v_1 \ldots v_n$, where $v_j \in V$ is a unique identifier for an activity
\item an extended set $W = w_0, \ldots w_{n+1}$ of the activities set $V$, where $\forall v_j \in V (w_j = v_j)$ with a unique dummy beginning activity $w_0$ and a unique dummy terminating activity $w_{n+1}$
\item the durations, processing time of the activities $Dur(v_i)$, so that $Dur(v_k) \in \mathbb{N}$ denotes the duration of activity $v_k \in V$ in time periods of time horizon $T$
\item a set of constraints which needs to be satisfied $E$, such that $(v_i,v_j) \in E$ means that $v_i \in V$ precedes activity $v_j \in V$ and $v_i \neq v_j$.
\item a precedence graph of the constraints and activities $G = (V, E)$, where we assume that the graph doesn't contain a cycle, otherwise the precedence relations are inconsistent.
\item a set of available resources $R$ where each element $r_i \in R$ is a unique identifier for a resource, split in renewable resources $R^{\rho}$ and  consumable (nonrenewable) resources $R^V$ \TODO{I believe this was notated as $R^\nu$} but with a possible overlap of doubly constrained resources ($r_j \in \renres \cap \conres$) \TODO{Dit lijkt me niet zo handig, stel $r_1$ is zo'n dubbele, wat is $Cap(r_1)$ dan?}
\item the availabilities of the resources $Av(R_i, t)$, so that $Av(R_k, t) \in \mathbb{N}$ denotes the availability of resource $R_k \in R$ at a time period $t \in \mathbb{N}$ in the time horizon $t < T$ \TODO{Net was het toch kleine $r_i$?}
\item the per period usage for an activity $v_j$ of resource $r_k$ are denoted by $Usage(v_j, r_k) \in \mathbb{N}$, such that it represents the amount of resource $r_k \in R$ used per time period in time horizon of executing activity $v_j \in V$ \TODO{Weird}
\end{itemize}

Find the feasible schedule where each activity gets executed with the available resources, the resources don't exceed their availability and all the constraints between the activities are satisfied where $Start(v_j) \in \mathbb{N}$ denotes the start time in periods of the time horizon $T$ for activity $v_j \in V$ and
$S = (Start(v_1) \ldots Start(v_n))$ is a feasible schedule for the activities $V = (v_1 \ldots v_n)$.

\subsection{Prerequisites}
In this section, the general problem focused on in this paper is explained.
Projects composed of activities, resources and constraints are introduced and a unifying notation is presented.
First, some terminology is discussed followed by a problem definition.

\subsubsection{Activities}
Each project consists of a finite set of activities(also called jobs) denoted as $V = v_1 \ldots v_n$, where each element of $V$ is a unique identifier for an activity and $n \in \mathbb{N}$ is the total number of activities.
Each activity $v_i \in V$ has a duration(processing time) $Dur(v_i) \in \mathbb{N}$, the time needed to complete the activity, and a starting time $Start(v_i) \in \mathbb{N}$, the time at which the activity should be executed. The duration $Dur(v_i)$ and the starting time $Start(v_i)$ are both denoted as a time period in the time horizon $T$.

We also have an extended set of activities $W = w_0, \ldots w_{n+1}$, which extends the activity set $V$ with a unique dummy beginning activity $w_0$ and a unique dummy ending activity $w_{n+1}$. This set contains all the activities of $V \subset W$ such that $forall_{1 \leq j \leq n}(w_j = v_j)$.

\subsubsection{Resources}
An activity requires a certain amount of resources to be available upon execution.
For example, changing a tire might require two workmen and one car lift.
In general, there are two kinds of resources: renewable and consumable (also named nonrenewable) resources.
With a consumable resources (such as petrol), once a part of a resource is used, it will no longer be available.
This contrasts with renewable resources, which do not deplete when used.
For renewable resources, the amount of units available of resource $R_k$ at any given time is $Av(R^{\rho}_k)$ (for example the number of machines in a shop).
The total amount of units available for a nonrenewable resource $k$ is noted by $Av(R^{V}_k)$.
The set of all renewable resources within a project is indicated with $R^{\rho}$ and the set of renewable resources is indicated by $R^{V}$. 

Activities consume a certain amount of one or more resources.
The amount that an activity $j$ consumes of a resource $k$ during a unit of time is denoted by $Usage(v_j, R^{\rho}_k)$ for a renewable resource and   $Usage(v_j, R^V_k)$ for an nonrenewable resource.


\subsubsection{Constraints}
Apart from activities that consume resources, additional constraints can be added.
Two of these are precedence constraints and temporal constrains.
Some activities need to be executed before another activity can be executed (for example, a surface needs to be sanded before it can be painted).
Such activity precedence can be visualized using a so-called precedence graph $G = (V, E)$ in which every activity is represented by a node and directed edges between nodes $v_i$ and $v_j$ are added if an action $v_i$ needs to be executed before $v_j$.
This is denoted by $(v_i,v_j) \in E$, where $v_i, v_k \in V$.

Temporal constraints can be used to specify a minimum and maximum time lag between two activities $i$ and $j$.
The notation used in this paper is $MinLag(v_i, v_j)$ and $MaxLag(v_i, v_j)$ respectively the minimum and the maximum time lag.
\TODO{use math func. Latex}

\subsubsection{Schedule}
\TODO{what is time horizon?, define RCS, two types of scheduling?, don't use completion times}
The general idea of resource scheduling is to minimize the project execution time without over-utilizing the resources or violate project constraints like the constraints discussed above.
The project itself can also have a deadline, the total amount of time units before the project needs to be completed.
A mapping of a unit of time in the model and an actual timespan in the real world can be indicated by the Time Horizon $t=1,2,3,\ldots,T$ and corresponding time intervals $[t-1,t]$.
The generated schedule may not exceed the Time Horizon $T$, when the project itself has an deadline.

The RCS problem schedule solution is denoted by a schedule that consists of a vector of the starting times for all the activities $S =  (Start(v_1) \ldots Start(v_n))$.
There are two types of schedules, time-feasible schedules that fulfills all time constraints denoted as $S_T$ and resource-feasible schedules that fulfills all the resource constraints denoted by $S_R$.
The conjunction of the time-feasible and the resource-feasible schedule will give you a solution for the RCS problem, this is denoted by $S = S_R \cap S_T$.
We also define the completion times for the activities in $V$, this is denoted by $C = Stop(v_1) \ldots Stop(v_n)$ where $Stop(v_j)$ for each activity $v_j$ has to be less than the Time horizon $T$, this can be denoted as following $\forall_{v_j}(v_j < T)$.

\subsection{Problem Complexity}
\TODO{Reductie naar bestaand NP-compleet probleem. Dit is de reden van het paper.}
The RCS problem belongs to the class of problems that are \emph{NP-hard}. According to the \TODO{the?} complexity theory a problem belongs to the class \emph{NP-hard} if its decision version is \emph{NP-complete} \TODO{Een probleem zit in NPC als hij NP Hard is en in NP zit, ik vind deze redenatie een beetje achterstevoren}. 
That the decision version of the RCS problem belongs to the class \emph{NP-complete} is proven and thereby is proven that the RCS problem belongs to the class \emph{NP-hard}. 
\TODO{Ik zou dit stuk vervangen door een alinea die zegt in welke paper het bewijs gegeven is voor $RCPSP \in NP-Complete$ en een intuitief bewijsje geven voor waarom dat zo is}\\
The time and space complexity of the RCS problem depends on the used algorithms and approaches used to solve the problem \TODO{Is inherent aan het problem. Het gaat om de EXACTE opplossing die NP-compleet is en bepaalde ruimte gebruikt}. 
There are currently two type of algorithms which are used in most cases, the exact algorithms and the approximation algorithms.
\TODO{Explain the time and space complexity between the two different approaches and add reference to the prove}

\subsection{Problem Extension}
\TODO{Waarom extension, wat is het nut van deze paragraaf, als je het doet, bespreek het dan goed}

\TODO{REMOVE?}

The basic PS \TODO{Wtf PS?} problem will sometimes not suffice. Several extensions to the PS problem have been made of which some will be discussed here. 

\subsubsection{Multi-mode project scheduling}
Multi-mode project scheduling is an extension to the original PS problem.
In short it takes multiple ways to perform and activity into account.
The different 'modes' in which an activity can be performed have different processing times and resource requirements for that activity.
See \citet{herroelen05} for details.
The approach to scheduling that is discussed in this paper does not account for multi-mode activities.
Research into multi-mode scheduling with PCP for continuous planning has not been done to the best of the authors' knowledge. 

\subsubsection{Stochastic processing times}
Often times it will be hard to estimate processing times for activities in practice.
There is a big chance that the actual processing time will be  different from the estimate.
In depth research has been done on stochastic processing time for project scheduling to model this deviation as has been shown in \cite{brucker99} covering several important problems \TODO{Welke problemen, welke resultaten, het zegt niks wat er staat} introduced by using a stochastic distribution for activity processing times.

\subsection{Summary}
\TODO{REMOVE / MOVE to 2.1}
\begin{tabular}{l l}
\hline
$T$ & time horizon \\
$t$ \TODO{$t_i$?} &  index of a time period in the time horizon $T = (t_1 \ldots t_n)$\\
$V$ & set of activities $V = (v_1 \ldots v_n)$ \\
$n$ & number of original activities \\
$E$ & set of constraints \\
$R^{\rho}$ & consumable resources \\
$R^V$ & renewable resources \\
$Dur(v_i)$ & the duration of activity $v_i$ \TODO{Ik heb tot nu toe de term "processing time" aangehouden, welke doen we?}\\
$Av(R_i)$ & the availability of resource $R_i$ \\
$Usage(v_j, R_k)$ & the per period usage for activity $v_j$ of resource $R_k$ \\
$Start(v_i)$ & the start time of activity $v_i$ \\
$Stop(v_i)$ & the start time of activity $v_i$ \\
$MinLag(v_i, v_j)$ & the minimum time lag between activity $v_i$ and $v_k$ \\
$MaxLag(v_i, v_j)$ & the maximum time lag between activity $v_i$ and $v_k$ \\
$S$ & a schedule consisting of st
art times for the activities $S = (Start(v_1) \ldots Start(v_n))$ \\
$C$ & a vector of end times for the activities $C = (Stop(v_1) \ldots Stop(v_n))$ \\
\hline
\end{tabular}

\newpage

\section{Schedule construction using PCP}
\TODO{Bastiaan}

\TODO{Structuur: Waarom PCP? Er is een probleem => NP-compleet. Dus vorm van benadering. Oberveer het volgende: als we ons puur beperken tot het tijdsaspect danr educeert het naar STN probleem wat op te lossen is in poly-time. Kunnen we het probleem niet splitsen zodat we resource deel apart nemenv antijdsdeel? We kijken naar resourceprofiel en voegen constraints toe. We beginnen met STN verhaal en daarna bespreken we hoe we constraints moeten toevoege. Twee methoden: Conflict Free en ESTA verhaal. Conflict Free is feasable voor alle starttimes en ESTA alleen voor 1 starttime. Chaining => ESTA oplossing ombouwen tot een POS.}

Since the RCPS problem belongs to the class of \emph{NP-complete} problems, no algorithm capable of constructing an optimal solution in $O(n^k)$ is known (unless of course, $P=NP$). In order to devise a solution using polynomial time, an approximation algorithm needs the be used. In this section, the approach of Precedence Constraint Posting (PCP) will be discussed.

\subsection{Introduction}
When regarding the RCPS problem, the problem has got both a time-related as well as a resource-related aspect. 
The time-related aspect focusses on scheduling the activities with respect to the given temporal constraints. 
As will be discussed below, this time-aspect of the problem can be reduced to a \emph{Simple Temporal Problem} (STP) \TODO{Refer to paper} which can be solved in $O(n^3)$ time. 
The result of this STP is a schedule or a set of schedules for which needs to be verified if the resource constraints are met. 
If there are any resource violations, \emph{Precedence Constraint Posting} (PCP) will offer a method of adding constraints to the STP in such a way that forces the resource requirements to be met in the generated schedule. 

The two aspects of this problem are regarded as two different layers, the \emph{Time Layer} and the \emph{Resource Layer}.
The Time Layer will first be discussed along with the reduction to the aforementioned STP. 
This STP will then be used to explain the relevance of the Resource Layer.

\subsection{Time Layer}
The temporal aspect of the problems deals with activity durations and precedence constraints.
These relations can be represented as a graph $G=(V,E)$ as defined in section \TODO{Reference near definition}. 
This so-called \emph{precedence graph} can be reduced to a \emph{Simple Temporal Network} (STN), which forms the basis for our STP.

\subsubsection{Simple Temporal Networks}
An STN is used to represent the temporal aspect of the RCSP and forms the basis for the STP.
An STN consists of a set $P$ of so-called \emph{time points} with associated time windows.
Relations between each pair of time points can be specified.
In an STN with uncertainty, the goal is to provide each time point $t_i$ with a lower and an upper bound such that, if a value for each $t_i$ is chosen in that interval, all the relations are satisfied.

Relations between time points can be added in the following way:
\begin{itemize}
\item $t_i \xrightarrow{[d_{ij},D_{ij}]} t_j$ called a \emph{free constraint}, meaning that the time $\delta$ between the two time points $t_i$ and $t_j$ is allowed to be scheduled within the range of $[d_{ij},D_{ij}]$.
\end{itemize}

\subsubsection{Reduction to STN}
\begin{enumerate}
\item Let S be an empty graph
\item For each activity $v_i \in V$, add two nodes (time points) $s_i$ and $e_i$ to S
\item Connect each pair of points $s_i$ and $e_i$ with a free constraint with bounds $[Dur(v_i), Dur(v_i)]$
\item For each precedence constraint $(v_i, v_j) \in E$, add a free constraint with bounds $[0, \infty]$
\end{enumerate}

The constraint introduced in step $3$ forces the duration of the activity to be exactly $Dur(v_i)$. The constraints added in step $4$ can have any, non-negative value which enforces the activities to take place some time after each other: a precedence constraint.

\subsubsection{Relations Temporal Constraints}
In the models used to simulate the problem at hand, it is convenient to express all constraints as temporal constraints.
A precedence constraint can be modeled as a temporal constraint with a minimum lag of $0$, and a maximum lag of $\infty$.

\subsubsection{Reduction to Temporal Constraints}
The temporal aspect of the problem can be modeled using \emph{time points}, \emph{free constraints} and \emph{contingent constraints} so as to form an STN \TODO{STNU?, de Uncertainty komt van die contigent constraints toch?} in the following way:
\begin{enumerate}
\item For each activity $v_i$, introduce two time points $s_i$ and $e_i$;
\item Add a contingent constraint $s_i \xrightarrow{[d_{ij}:D_{ij}]} e_i$ for each activity; \TODO{Zitten de $i$ en de $j$ hier wel goed?}
\item Add a free constraint $s_i \xrightarrow{[d_{ij},D_{ij}]} e_i$ for each temporal constraint in the problem definition.
\end{enumerate}

\TODO{insert hip plaatje van een STNU hier}

As defined above, each activity is represented by two time points.
The contingent constraint connecting the two time points represents the duration of the activity.
The exact processing time of the activity is not yet known at this stage, only a worst-case and a best-case processing time are given.
These best-case and worst-case processing times are used respectively as lower and upper bounds for the contingent constraint.
The schedule needs to be generated in such a way that for each possible value for all of the contingent constraints (all possible combinations of processing times), the free constraints are met.

Using a technique called \emph{constraint propagation} \cite{policella07}, it is possible to bound the domains of each time point $t_i$ such that $t_i \in [lb_i, ub_i]$, in which $lb_i$ represents the lower bound and $ub_i$ represents the upper bound of the time domain for the time point $t_i$, in such a way that all relations are satisfied regardless of which duration the activities have.
It has been proven that it is possible to find such bounds for $n$ constraints in $O(n^3)$ time.
Note that the resulting STNU represents a set of schedules rather then one schedule in particular, because the exact execution times are not known yet.
Since a solution assigns values (instead of bounds) to each time point, in order to obtain a solution, a value needs to be given to each time point $t_i$.
One could, for example, take the lower bound of each time point and use that as the value for each time point.
\TODO{... to create an actual schedule instance?}

\subsection{Resource Layer}
\TODO{Verhaal van Ciesta}
\subsubsection{Resource leveling}
\subsubsection{Minimum Conflict Sets}

\subsection{Precedence Constraint Posting}
\TODO{Explain how the two layers can be brought together}
\subsubsection{Introduction}
In the classical RCPS problem described in the previous section, the input parameters of the problem, such as processing times of activities, need to be known before generating the schedule.
The solution of this problem is a fixed schedule which can then be employed in practice.
However, this off-line generation of the schedule does not offer the amount of flexibility that is often desired in practical problems.
For example, the processing time of the activities might not be precisely known beforehand and only best-case and a worst-case times can be given.
A schedule that is able to react to uncertainties and can be updated once more and more parameters become known is desired.
Using Precedence Constraint Posting (PCP), the solution of the problem is a flexible augmented schedule that can be generated using an estimate of the input parameters and can then be updated in real-time (also called on-line) to adapt to the actual values.
\TODO{Complex, long, this/that sentence :p}

\subsubsection{Scheduling Policy}
\subsubsection{Partial Ordering Schedules \TODO{Order?}}

\subsubsection{Problem Extension}
Instead of assuming each activity to have a fixed processing time, each activity is linked to a worst-case or lower bound ($p_{j} \in \mathbb{N}$) and a best-case or upper bound ($P_{j} \in \mathbb{N}$) processing time.
Whereas the classical solution of the RCPS problem was given in the form of one schedule, the result of PCP is an augmented precedence graph that defines a set of possible schedules.
This graph can be used to devise a starting time for activity $a_i$ in real-time, based on the execution history so far.
The graph needs to be constructed in such a way that 
\TODO{I don't think this paragraph is in line with the current document structure}

\newpage

\section{Continuous scheduling}
\TODO{MickL}

\TODO{rewrite, remove planning, continuous scheduling, use examples}

Resource constraint scheduling algorithms focus on solving a given set of tasks, time and resource constraints in a most efficient schedule.
This makes these solutions quite effective on more complex, closed sets of tasks, but the created schedules are not adaptive when the resources or set change during execution.
In many practical environments of the scheduling problem there is another factor to take into account: continuity.
In these applications the set of resources and tasks is not closed, but tasks are added during the process.
This requires the schedule to be changed or updated while processing the tasks in the schedule.

We will first discuss schedule repair solutions, which take into account that some changes or problems may occur during schedule execution.
These repair solutions provide a way to repair a schedule when certain small changes occur, for example due to delays in the process.
Then we will review partial order schedules, an example of a solution which provides a flexible schedule.
The partial order schedule contains several feasible schedules, enabling it to adjust to changes without the need for repair.
\TODO{}Precedence constraint posting can be used to deliver partial order schedules.
Finally we will take a look at how these repairs can be used to create a flexible and robust schedule, that can handle the addition of new tasks during the process.

\subsection{Schedule repair}
One solution to continuous changes to the problem set is schedule repair.
This reactive solution tries to repair the schedule as soon as problems or conflicts are detected.
Using this type of repair requires the processing of the schedule to be paused, a new schedule to be calculated based on the changed problem set and the execution to be restarted with the new schedule.
Pausing and repairing a schedule is a very time consuming solution, especially in environments where many changes may occur.
To maximize effectiveness, the repair needs to be fast and allow the execution to continue with the minimum delay.
Different solutions exist, which mostly favor either a quick repair with less accuracy or a slower complete repair.
\cite{policella07}

Schedule repair gives the robustness to build a schedule although there are unknown factors in the environment, but requires the execution to be paused and repaired when problems occur.

\subsection{Flexible schedules}
Another solution is a more proactive approach, using flexible schedules.
Flexible schedules are schedules which retain flexibility and can cope with some changes without the need for rescheduling.
These solutions often present a form of redundancy, in time and/or resources.
This redundancy represents multiple schedules within one schedule.
Therefore allowing to select the execution which most resembles the actual environment.
In flexible scheduling it is required to know all types of changes that may occur in advance.
This requirement poses a big strain on the planning on forehand.
If an unexpected change should occur, the schedule might not be capable of handling that change.
In this case rescheduling is necessary.

Flexible scheduling provide a way to use the original flexible schedule and adopt it to the environment, but can only cope with known changes and problems.

\subsection{Partial order schedules}
\TODO{Proactive, output is hetzelfde als input}
Partial order schedules are solutions which take the flexibility a step further.
In partial order schedules, as defined in \citet{policella07}, each activity retains a set of feasible start times.
In this manner a set of possible schedules is available, rather than just one solution.
In case there is a change, which is not too big, a simple modification to the schedule suffices to generate a new feasible solution.
Partial order scheduling uses repairs, just as the iterative repair schedules that were previously mentioned.
However, the partial order schedule already contains several options for an easy fix of the schedule.
Therefore the time consumed in the recalculating of the schedule is reduced tremendously.

There is a possibility of the changes being significant enough to force a complete rescheduling, which takes more time than the quick repair.
The partial order schedules are quicker to repair, but require more time initially and when large changes occur.
In addition to this problem, the quick repairs do not represent an optimal schedule.
Therefore partial order schedules are a better and more flexible manner of continuous planning, but still does not result in the best solution.

\subsection{Precedence constraint posting in partial order schedules}
\TODO{see policella07 chapt 5}

\TODO{MickG}

\TODO{REMOVE / MERGE with previous chapter}
Precedence Constraint Posting (PCP) is a technique for finding feasible schedules by adding new precedence constraints to the original project graph in such a way that start times can be computed for whatever combination of activity durations without compromising the resource constraints.
We will first take a look at how PCP works and then describe how this technique can be utilized in continuous planning.


The PCP approach to scheduling is suitable for continuous planning.
A lot of the methods for PCP create a POS in the form of Simple Temporal Network.
This network has temporal constraints that ensure not only consistency of the activity precedence relations but also that resource capacities are not exceeded.
An advantage of using STNs is that it is possible to post additional constraints during schedule executions to recalculate a new feasible schedule from the POS in polynomial time.

If for example a project manager notices that a certain activity takes longer than expected, the scheduler can post an additional temporal constraint to the POS and try to create a new schedule from it.
Notice that the new schedule will probably not be optimal and in some cases it might not be possible to create a new resource and time feasible schedule.
In that case a new POS has to be created from the remaining project graph.


A complete recalculation will not be necessary as frequent as with other techniques because the POS can be utilized in some cases of, possibly unexpected, lowered resource capacity, extended activity processing time or additional activities.


\newpage

\section{Conclusion}
\TODO{The overall conclusion.
Huidige tekst verplaatst uit continuous}
Continuous planning poses a special problem in the scheduling domain.
Although continuous planning seems obvious and the presence is ubiquitous in practical applications, there is no effective theoretical solution capable of providing a optimal solution in continuous planning.
There is always a trade-of between optimal solutions and processing speed.
Often the pressure to provide a quick solution overrules the optimum option.

\newpage
\bibliographystyle{plainnat}
\bibliography{references}

\end{document}








