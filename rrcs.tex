\documentclass{article}
\usepackage{a4wide}
\usepackage{amssymb}

\newcommand{\leading}[1]{\textbf{#1}}
\usepackage{xcolor}

%% if your are not using LaTeX2e use instead
%% \documentstyle[bnaic]{article}

%% begin document with title, author and affiliations

\title{Precedence Constraint Posting in Continuous Planning; Practical Adoption of the Scheduling Problem.}
\author{M. van Gelderen  \and
    R.M. de Lange \and
    B. Gris\`el \and
    F. van Tienen}

\pagestyle{empty}

\newcommand{\TODO}[1]{{\color{red}\textbf{TODO: #1}}}

\begin{document}
\maketitle
\thispagestyle{empty}

\begin{abstract}
\TODO{abstract als laatste schrijven}
\end{abstract}


\section{Introduction}

The resource scheduling problem is a problem concerned with the scheduling of tasks on a specific set of resources.
These, often scarce, resources form a constraint on the volume of tasks which can be processed at any one time.
Scheduling focuses on optimizing a schedule to either consume the least amount of resources, or perform all tasks in minimum required time.  \cite{brucker99}
There are several different forms of resource constrained scheduling, all dealing with a specific type of scheduling problem.

The main aspects that will be covered in this paper are precedence constraint posting and continuous planning.
Precedence constraints require a set of tasks to be performed in a specific order, requiring one task to be performed before the other.
Also, temporal constraints have a part in this scheduling problem, but this will not be the focus in this paper.
Continuous planning is a form of scheduling where the scheduling process is not completed, the process continuous while schedule is being executed.
In this case the set of tasks to be performed is not closed, meaning that during execution of the schedule the set of tasks may change and with it the optimal solution.
However, it may also be undesirable to introduce major changes in the baseline schedule.
Therefore, the continuous planning aspect of scheduling is one of the more complex issues and also a quite common phenomenon in practical situations.

In this paper we will combine these to elements in the scheduling problem and study this in the light of robust resource constrained scheduling.
Robust resource constrained scheduling requires that the schedule not only gives an efficient solution to perform all tasks.
It  should also take in account that some jobs might be delayed or resources become unavailable due to circumstances and then still give an efficient solution to the problem.
These kinds of changes can also occur due to the continuous nature of the application in which the scheduling is applied, for example by adding new tasks to the set.

In the following chapter we will be dealing with the theory on scheduling problems and discus several relevant research papers that cover the topic.
After which the two main aspects, precedence constraint posting and continuous planning, will be discussed further and then combined and linked to practical problems.
We will take a deeper look at the practical applications for the theory, with their respective overlap and conflicts with the aforementioned theory.
As one might expect, several issues occur in day-to-day applications of scheduling and not every aspect that is found in practice is covered by theoretical research on this subject.
As is not unusual for mathematical or logical models of real-life problems, there is aways some assumption on preconditions which needs to be done to make a simplified model.

This paper is intended to link the practical applications of the scheduling problem to several aspects of the theory, thereby addressing the incompleteness of the problem theory and discover what aspects need further research.


\newpage

\section{Problem background}

Different aspects of the underlying theory leading to RRCS.  Per point explanation of the theory with references to relevant papers. This will be the main section of the paper.

\TODO{Describe what notation conventions we will adopt. I suggest using the notation from P. Brucker et al, Eauropean Journal of Operational Research 112 (1999). }

\subsection{Prerequisites}
In this section, the general problem focussed on in this paper is explained. Projects composed of activities, resources and constraints are introduced and a unifying notation is presented. First, some terminology is discussed followed by a problem definition.

\subsubsection{Activities}
Each project consists of a finite set of activities (also called jobs). Each activity has a processing time $p_{j} \in \mathbb{N}$; the time needed to complete the activity. The set of all activities is $V$.

\subsubsection{Resources}
An activity requires a certain amount of resources to be available upon execution. For example, changing a tyre might require two workman and one car lift. In general, there are two kinds of resources: renewable and nonrenewable resources. With a nonrenewable resources (such as petrol), once a part of a resource is used, it will no longer be available. This contrasts with renewable resources, which do not deplete when used. For renewable resources, the amount of units available of resource $k$ at any given time is $\mathcal{R}^{\rho}_k$ (for example the number of machines in a shop). The total amount of units available for a nonrenewable resource $k$ is noted by $\mathcal{R}^{v}_k$. The set of all renewable resources within a project is indicated with $\mathcal{R}^{\rho}$ and the set of renewable resources is indicated by $\mathcal{R}^{v}$. 

Activities consume a certain amount of one or more resources. The amount that an activity $j$ consumes of a resource $k$ during a unit of time is denoted by $r^{\rho}_{jk}$ for a renewable resource and  $r^{v}_{jk}$ for an nonrenewable resource. 

\subsubsection{Contraints}
Apart from activities that consume resources, additional constraints can be added. Two of these are precedence constraints and temporal constrains. Some activities need to be executed before another activity can be executed (for example, a surface needs to be sanded before it can be painted). Such activity precedence can be visualized using a so-called precedence graph $(V, E)$ in which every activity is represented by a node and directed edges between nodes $A_i$ and $A_j$ are added if an action $A_i$ needs to be executed before $A_j$.

Temporal constraints can be used to specify a minimum and maximum time lag between two activities $i$ and $j$. The notation used in this paper is $d^{min}_{ij}$ and $d^{max}_{ij}$ respectively.

\subsubsection{Schedule}
The general idea of resource scheduling is to minimize the project execution time without over-utilising the resources or violate project constraints like the constraints discussed above. The project itself can also have a deadline, the total amount of time units before the project needs to be completed. A mapping of a unit of time in the model and an actual timespan in the real world can be indicated by the Time Horizon $t=1,2,3,. . .,T$ and corresponding time intervals $[t-1,t]$. 

\subsection{STN}
\TODO{Wat is dit: Bastiaan.
Mick G: Ik gok Simple Temporal Networks}

\subsection{Formal Definition}
\TODO{Freek, naar voren halen}
The general definition for the robust resource constraint scheduling problem is defined by a tuple $(V, p, E,R,B, b)$:\\

\cite{brucker99}

Given:
\begin{itemize}
\item a set of activities $V$
\item the durations of the activities $(p_1\ldots p_n)$
\item a set of constraints which needs to be satisfied $E$
\item a set of available resources $\mathcal{R}$
\item the availabilities of the resources $B$
\item the demands for an activity $j$ of resource $k$ $(r_{11}\ldots r_{jk})$
\end{itemize}

\TODO{Explain what the question of the resource constraint scheduling is}

\subsection{Problem Complexity}
\TODO{Freek}
\subsubsection{Problem Classification}
RRCS belongs to the class of problems that are $NP-hard$. According to the complexity theory a problem belongs to the class $NP-hard$ if its decision version is $NP-complete$. We will use this theory to prove that RRCS belongs to the class $NP-hard$.

\TODO{Volgens mij kunnen we hier volstaan met een verwijzing naar een paper waar dit reeds gedaan is. Volgens mij kunnen de subkopjes dan ook weg.}
\subsubsection{Time Complexity}
\subsubsection{Space Complexity}

\subsection{Problem Extension}
The basic PS problem will sometimes not suffice. Several extensions to the PS problem have been made of which some will be discussed here. 

\subsubsection{Multi-mode project scheduling}
Multi-mode project scheduling is an extension to the original PS problem. In short it takes multiple ways to perform and activity into account. The different 'modes' in which
\TODO{Reference Herroelen, Reactive scheduling in the multi-mode RCPSP}

\subsubsection{Stochastic processing times}
\TODO{Staat dit al ergens anders?}

\section{Constraint posting}
\TODO{Bastiaan, verdelen in subkpjes}

\section{Continuous planning} \TODO{Mick de LangeR}

The subject of continuous planning is somewhat neglected in classical scheduling problems and research.
With continuous scheduling one needs to taken in to account that the set of tasks is not complete or closed when a schedule is requested.
Therefore continuous planning requires more flexibility in the solution and multiple options of editing the solution.

\subsection{AI planning}
\cite{smith00}
\cite{laborie03}

\subsection{Maintenance scheduling}
Maintenance scheduling is very near to AI planning and the given continuous planning.

\subsection{Repairing schedules}
\cite{chien00}

\section{Contraint Posting in Continuous planning}
\TODO{Mick van Gelderen}

\subsection{Conclusion}
\TODO{Mick de Lange-R}

\section{Problem Practice}
This section will discuss how the practice problems can be dealt with using the theory described in the previous section. First, the problems that can be solved using this theory are discussed. After that, the problems which cannot be solved directly using the theory are explained.\\

\begin{itemize}
\item Project scheduling under uncertainty: Survey
and research potentials
\item A Comparison of Heuristic and Optimum Solutions
\item Reactive scheduling, improving the robustness of schedules
\end{itemize}

\subsection{Solvable problems with PCP}
\TODO{Freek}
\begin{itemize}
\item What are the examples from the real world? 
\item What are the underlying problems of these examples?
\item How can they be solved using the theory described above?
\end{itemize}

\subsection{Related problems with PCP}
\TODO{Bastiaan}
\begin{itemize}
\item What are the examples from the real world?
\item Trains, trucks, maintenance scheduling, etc.
\item Airport planning / different factors: maintenance / tanking / bagage / boarding
\item What are the underlying problems of these examples?
\item How does the above theory needs to be augmented in order to be able to solve the problems?
\end{itemize}

\section{Conclusion}
\TODO{The overall conclusion.}

\bibliographystyle{plain}
\bibliography{references}

\end{document}








