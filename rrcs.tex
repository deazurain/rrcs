\documentclass{article}
\usepackage{a4wide}
\usepackage{amssymb}
\usepackage{amsmath}
\usepackage[numbers]{natbib}
\usepackage{xcolor}
\usepackage[colorlinks,linkcolor=blue,urlcolor=blue,citecolor=blue]{hyperref}
\usepackage{mathrsfs}
\usepackage{comment}

\usepackage{tikz}
\usetikzlibrary{calc,positioning,shapes,arrows,decorations.pathreplacing}
\usetikzlibrary{shapes.multipart}


\title{Resource Constraint Scheduling in Dynamic Applications using Precedence Constraint Posting.}
\author{M. van Gelderen  \and
    R.M. de Lange \and
    B. Gris\`el \and
    F. van Tienen}
\date{}

\pagestyle{empty}

\newcommand{\TODO}[1]{{\color{red}\textbf{TODO: #1}}}

\newcommand{\renres}[0]{R^\rho} %renewable resource
\newcommand{\conres}[0]{R^\nu} %consumable resource
\newcommand{\av}[1]{\textit{Av}(#1)} %availability
\newcommand{\dur}[1]{\textit{Dur}(#1)} %durability
\newcommand{\usage}[1]{\textit{Usage}(#1)} %usage
\newcommand{\start}[1]{\textit{Start}(#1)} %start time

\setlength{\parindent}{0pt}
\setlength{\parskip}{\baselineskip}

\begin{document}
\maketitle
\thispagestyle{empty}

\begin{abstract}
This paper gives an overview of the application of the precedence constraint posting technique in the resource constraint scheduling problem.
A unifying notation will be introduced and the procedure for generating a feasible schedule will be discussed. Precedence constraint posting proves to be a convenient method to use in an environment in which flexibility, robustness and continuity are of major importance.
Using precedence constraint posting, a solution for the resource constraint scheduling problem can be found in polynomial time and executing small updates in the generated schedule can therefore also be done in polynomial time.
The exact methods to do so are outlined, from iterative algorithms to partial ordering schedules. Finally, we will explain how and why precedence constraint posting is suited for use in continuous scheduling problems. 
\end{abstract}

\newpage

\section{Scope and purpose}
%We bekijken precedence constraint posting als methode voor het genereren van een schema voor het continuous resource constrained project scheduling probleem. 
Scheduling is a common problem in many day-to-day applications, but represents one of the more complex problems to solve.
Scheduling is the process of assigning start times to a set of activities, in other work also referred to as tasks or jobs, in such a way that a number of conditions are met. 
In many applications several complicating factors may arise.
Scheduling can deal with some of these complicating factors using constraints.
For instance, a limited number of resources, uncertain length of task execution and deadlines.
These are examples of relatively simple factors. 
More complicating factors are those that can be found in practical applications, like: break-down of resources, delays or uncertain number of tasks.
Especially in dynamic environments, which is the type of environment many practical applications need to deal with.
In this paper we will explore the resource constraint scheduling problem and explain how the theory can be used to cope with these more complicating factors in a dynamic environment.

%Het doel van de paper is om een volledige informatie bron zijn voor een 3e jaars TI student over het hierboven genoemde
The purpose of this paper is to inform fellow students on the topic of resource constraint scheduling in dynamic applications.
The paper aims to be a good source of information on the topic of resource constraint scheduling and its link to the practical applications seen in daily life.
Although the reader is presumed to have prerequisite knowledge about for example set theory, graph theory, algorithmics and complexity theory, most in-depth information aims to be complete and clear.

%Eerst geven we een intuïtieve definitie van het RCPSP. Dan volgt een formele definitie van RCPSP. 
First we will give an intuitive definition of resource constraint project scheduling. 
It sketches the problem and gives examples of constraints that are encountered. 
This is followed by a formal definition of the problem along with the notation used throughout this paper. 
%Hierna volgt een intuïtieve definitie van STNs, daarna volgt een formele definitie van STNs. 
We will then explain simple temporal networks. 
Knowledge of STNs is required by the precedence constraint posting approach. Instances of RCPSP can be represented by a STN. 
Temporal networks are a simple and flexible way to represent schedules.
The formal definition of STNs is given afterwards. 


%Dan kunnen we de diepte in en leggen we uit wat precedence constraint posting is.
Precedence constraint posting and partial order schedules will be explained in section \ref{section:PCP}. 
We show how PCP utilizes simple temporal networks to represent a RCPSP instance and how PCP utilizes the flexibility of such networks so that the schedule can take new constraints into account. 

%Dan maken we de transitie naar continuous planning en waarom het nodig is.
Section \ref{section:continuous} elaborates on the obstacles encountered due to the continuity of real life problems. 
The problems that are introduced because of this factor are explained. 
Scheduling methods addressing this issue by for example improving the flexibility of schedules will be discussed as well.
Making schedules flexible is done in two different approaches: reactive and proactive.
Reactive means that a schedule is created in advance and when problems occur, it is rescheduled to deliver a new schedule.
Proactive holds that a schedule is created which itself can handle some changes.
These schedules mostly contain multiple solutions in one schedule.

%Hierna leggen we uit waarom PCP geschikt is om continuous planning mee te realiseren.
The last section will combine the resource constraint scheduling, partial order schedule and precedence constraint posting subjects to show that a flexible and robust schedule can be created using these theories. 

\newpage

\section{Problem background}
Before we start we need to explain a bit more about what the RCPS problem is and how we define it.
In this section we will start with the formal definition of our RCPS problem, where our unifying representation is presented.
Then we will try to explain the problem in some more depth by explaining a bit more about our prerequisites that are used in the formal definition.
As last we will try to explain the problem complexity of the RCPS and how this is proven.


\subsection{Formal Definition}
The general definition for the resource constraint scheduling problem (RCPSP) is defined in the following definition\cite{brucker99}.

Given:
\begin{itemize}
\item a horizon $T$ that denotes the total time horizon for the problem, where $T \in \mathbb{N}$
\item a set of activities (also called jobs or tasks) $V = v_1 \ldots v_n$, where $v_j \in V$ is a unique identifier for an activity
\item an extended set $W = w_0, \ldots w_{n+1}$ of the activities set $V$, where $\forall_{1 \leq j \leq n} (w_j = v_j)$ with a unique dummy beginning activity $w_0$ and a unique dummy terminating activity $w_{n+1}$
\item the durations, processing time of the activities $\dur{v_i}$, so that $\dur{v_k} \in \mathbb{N}$ denotes the duration of activity $v_k \in V$ in time periods of time horizon $T$
\item a set of constraints which needs to be satisfied $E$, such that $(v_i,v_j) \in E$ means that $v_i \in V$ precedes activity $v_j \in V$ and $v_i \neq v_j$.
\item a precedence graph of the constraints and activities $G = (V, E)$, where we assume that the graph doesn't contain a cycle, otherwise the precedence relations are inconsistent.
\item a set of available resources $R$ where each element $r_i \in R$ is a unique identifier for a resource, split in renewable resources $\renres$ and  consumable (nonrenewable) resources $\conres$.
\item the availabilities of the resources $\av{r_i, t}$, so that $\av{r_k, t} \in \mathbb{N}$ denotes the availability of resource $r_k \in R$ at a time period $t \in \mathbb{N}$ in the time horizon $t < T$
\item the per period usage for an activity $v_j$ of resource $r_k$ is denoted by $\usage{v_j, r_k} \in \mathbb{N}$, such that it represents the amount of resource $r_k \in R$ used per time period in time horizon of executing activity $v_j \in V$
\end{itemize}

Find the feasible schedule where each activity gets executed with the available resources, the resources don't exceed their availability and all the constraints between the activities are satisfied where $\start{v_j} \in \mathbb{N}$ denotes the start time in periods of the time horizon $T$ for activity $v_j \in V$ and
$S = (\start{v_1} \ldots \start{v_n})$ is a feasible schedule for the activities $V = (v_1 \ldots v_n)$.

\subsection{Prerequisites}
As shown in the formal definition the problem is composed of activities, resources and constraints.
In this section we will discuss what activities, resources and constraints really are in the RCPS problem.

\subsubsection{Activities}
Each project consists of a finite set of activities (also called jobs) denoted as $V = v_1 \ldots v_n$, where each element of $V$ is a unique identifier for an activity and $n \in \mathbb{N}$ is the total number of activities.
Each activity $v_i \in V$ has a duration(processing time) $\dur{v_i} \in \mathbb{N}$, the time needed to complete the activity, and a starting time $\start{v_i} \in \mathbb{N}$, the time at which the activity should be executed. The duration $\dur{v_i}$ and the starting time $\start{v_i}$ are both denoted as a time period in the time horizon $T$.

We also have an extended set of activities $W = w_0, \ldots w_{n+1}$, which extends the activity set $V$ with a unique dummy beginning activity $w_0$ and a unique dummy ending activity $w_{n+1}$. This set contains all the activities of $V \subset W$ such that $forall_{1 \leq j \leq n}(w_j = v_j)$.

\begin{figure}[h]
	\centering
	% Author: Mick van Gelderen
% Created from http://www.texample.net/tikz/examples/class-diagram/ and other examples and resources. 

%\documentclass{standalone}
%\usepackage{tikz}
\usetikzlibrary{calc,positioning,shapes,arrows,decorations.pathreplacing}
\usetikzlibrary{shapes.multipart}


%\begin{document}

\tikzstyle{activity}=[rectangle, draw=black, text centered, text=black, thick, minimum height=1.8em, minimum width=1.8em]
\tikzstyle{dummy}=[rectangle, fill=black!7, draw=black, text centered, text=black, thick, minimum height=1.8em, minimum width=1.8em]
\tikzstyle{precedence}=[->,>=stealth, draw=black!70, thick]

\newcommand{\activity}[3]{\node (v#1) [activity, text width=#2cm, #3] {$v_#1$};}

\begin{tikzpicture}[node distance=.8cm]
		\node (v0) [dummy] {$v_0$};
    \activity{2}{2}{right=1.6cm of v0}
    \activity{1}{3}{above=of v2}
    \activity{3}{2}{right=of v1}
    \activity{4}{1}{below=of v3}
    \activity{5}{5}{below=of v2}
    \activity{6}{1}{right=1.6cm of v4}
		\node (v7) [dummy, right=of v6] {$v_7$};
		
    \draw[precedence] (v0.east) to[out=0,in=180] (v1.west);
    \draw[precedence] (v0.east) to[out=0,in=180] (v2.west);
    \draw[precedence] (v0.east) to[out=0,in=180] (v5.west);
    \draw[precedence] (v1.east) to[out=0,in=180] (v3.west);
    \draw[precedence] (v2.east) to[out=0,in=180] (v4.west);
    \draw[precedence] (v3.east) to[out=0,in=180] (v6.west);
    \draw[precedence] (v4.east) to[out=0,in=180] (v6.west);
    % start bending from the projection v4 on the line y=v3.y
    \draw[precedence] (v5.east) -- ($(v5)!(v4)!($(v5)+(1,0)$)$) to[out=0,in=180] (v6.west);
		\draw[precedence] (v6.east) to[out=0,in=180] (v7.west);		    

    % Descriptions
    \path (v3.east) to[out=0,in=180] coordinate (temp) (v6.west);
    \draw[shorten <=2pt, shorten >=2pt] (temp) -- ++(.8,.6) node[anchor=west, yshift=2pt, text width=2cm] {Precedence constraint $(v_3, v_6) \in E$};
    
    \draw[shorten <=2pt, shorten >=2pt] (v5.south) -- ++(.8,-.6) node[anchor=west, yshift=-2pt] {Activity $v_5$};
    
    % Funky braces
    \draw[decorate,decoration={brace,amplitude=8pt}]
        let \p1 = (v5.west), \p2 = (v6.east), \p3 = (v1) in 
        (\x1, \y3+1cm) -- (\x2, \y3+1cm) node[midway, above=10pt] {$V = \{v_1, \ldots, v_6\}$};
        
    \draw[decorate,decoration={brace,amplitude=8pt}]
        let \p1 = (v0.west), \p2 = (v7.east), \p3 = (v1) in 
        (\x1, \y3+2cm) -- (\x2, \y3+2cm) node[midway, above=10pt] {$W = V \cup \{v_0, v_7\}$};
    
\end{tikzpicture}

%\end{document}
	\caption{Example of an activity graph with dummy activities. }
	\label{fig:activity_graph}
\end{figure}

\subsubsection{Constraints}
Apart from activities that consume resources, additional constraints can be added.
Two of these are precedence constraints and temporal constrains.
Some activities need to be executed before another activity can be executed (for example, a surface needs to be sanded before it can be painted).
These dependencies are specified by a set $E \subset V \times V$.
This means that a precedence relation between activities $v_i$ and $v_j$, where $v_j$ can be started only after $v_i$ is finished, is represented as $(v_i, v_j) \in E$. 
Figure \ref{fig:activity_graph} shows an augmented graph a visualisation of the constraint and activity sets.

\subsubsection{Resources}
An activity requires a certain amount of resources to be available upon execution, this can either be a renewable resource or a consumable (also called nonrenewable) resource.
With consumable we define resources, that when they are used it will no longer be available.
This contrasts with renewable resources, which do not deplete when used.
For example a renewable resource could be the workmen and a car lift that are required to change a flat tire and the consumable resource could be the replace tire itself.

For renewable resources, the amount of units available of resource $r_k \in R$ at any given time is $\av{\renres_k, t} \in \mathbb{N}$ (for example the number of machines in a shop).
The total amount of units available for a nonrenewable resource $r_k \in R$ is noted by $\av{\conres_k, t} \in \mathbb{N}$.
Both return a number which represents the availability in units.
The set of all renewable resources within a project is indicated with $\renres$ and the set of renewable resources is indicated by $\conres$. 

Activities consume a certain amount of one or more resources.
The amount that an activity $v_j \in V$ consumes of a resource $r_k \in R$ during a unit of time is denoted by $\usage{v_j, \renres_k} \in \mathbb{N}$ for a renewable resource and   $\usage{v_j, \conres_k} \in \mathbb{N}$ for an nonrenewable resource.
Both return a number which represents the usage of a resource in the same type of units as the availability.


\subsubsection{Constraints}
Apart from activities that consume resources, additional constraints can be added.
Some activities need to be executed before another activity can be executed (for example, a surface needs to be sanded before it can be painted).
Such activity precedence can be visualized using a so-called precedence graph $G = (V, E)$ in which every activity is represented by a node and directed edges between nodes $v_i$ and $v_j$ are added if an action $v_i$ needs to be executed before $v_j$.
This is denoted by $(v_i,v_j) \in E$, where $v_i, v_k \in V$.

\subsubsection{Schedule}
The general idea of resource scheduling is to minimize the project execution time without over-utilizing the resources or violate project constraints like the constraints discussed above.
The project itself can also have a deadline, the total amount of time units before the project needs to be completed.
A mapping of a unit of time in the model and an actual timespan in the real world can be indicated by the Time Horizon $t=1,2,3,\ldots,T$ and corresponding time intervals $[t-1,t]$.
The generated schedule may not exceed the Time Horizon $T$, when the project itself has an deadline.

\begin{figure}[h]
	\centering
	% Author: Mick van Gelderen
% Created from http://www.texample.net/tikz/examples/class-diagram/ and other examples and resources. 

%\documentclass{standalone}
%\usepackage{tikz}
\usetikzlibrary{calc,positioning,shapes,arrows,decorations.pathreplacing}
\usetikzlibrary{shapes.multipart}


%\begin{document}

\tikzstyle{activity}=[rectangle, draw=black, text centered, anchor=south west, text=black, minimum height=1cm, minimum width=1cm]
\tikzstyle{axis}=[->,thick,>=stealth]

\begin{tikzpicture}[node distance=1]
    \draw[axis] (-.5,0) -- (8.5,0) node[anchor=north] {Time};
    \draw[axis] (0,-.5) -- (0,2.5) node[anchor=north east] {Activities} ;
    
    \node (v0) [activity] at (0,0) {$v_0$};
    \node (v2) [activity, minimum width=4cm] at (1,0) {$v_2$};
    \node (v1) [activity, minimum width=3cm] at (1,1) {$v_1$};
    \node (v3) [activity, minimum width=2cm] at (4,1) {$v_3$};
    \node (v4) [activity, minimum width=2cm] at (5,0) {$v_4$};
    \node (v5) [activity] at (7,0) {$v_5$};
    
    \draw[dotted] (v1.south east) -- ++(0,-1) node[anchor=north] {$t_3$};
    \draw[dotted] (v3.south east) -- ++(0,-1) node[anchor=north] {$t_5$};
    \node [anchor=north] at (v2.south west) {$t_0$};
    \node [anchor=north] at (v4.south west) {$t_4$};
    \node [anchor=north] at (v5.south west) {$t_6$};
    
\end{tikzpicture}

%\end{document}
	\caption{\TODO{verzin caption}}
	\label{fig:time_schedule}
\end{figure}

The RCPS problem schedule solution is denoted by a schedule that consists of a vector of the starting times for all the activities $S =  (\start{v_1} \ldots \start{v_n})$.
There are two types of schedules, time-feasible schedules that fulfills all time constraints denoted as $S_T$ and resource-feasible schedules that fulfills all the resource constraints denoted by $S_R$.
The conjunction of the time-feasible and the resource-feasible schedule will give you a solution for the RCPS problem, this is denoted by $S = S_R \cap S_T$.
We also define the completion times for the activities in $V$, this is denoted by $C = Stop(v_1) \ldots Stop(v_n)$ where $Stop(v_j)$ for each activity $v_j$ has to be less than the Time horizon $T$, this can be denoted as following $\forall_{v_j}(v_j < T)$.

\subsection{Problem Complexity}
The RCPSP belongs to the class of problems that are \emph{NP-complete} which is the subset of problems that are in the class \emph{NP-hard} and \emph{NP}. 
According to the complexity theory a problem belongs to the class \emph{NP-complete} if the problem is in \emph{NP} and every problem in \emph{NP} is reducable to the problem.
This can be easily proven if you reduce the RCPS problem to an already existing \emph{NP-complete} problem, because then it automaticly is reducable to every problem in \emph{NP}.

\TODO{Reference a paper where this is proven}

\begin{comment}
\TODO{Dit moest er toch uit?}
\subsection{Problem Extension}
The basic RCPS problem will sometimes not suffice. Several extensions to the RCPS problem have been made to better suit real-life environments. The following paragraphs will provide a brief overview of how the problem can be augmented. Although it is not in the scope of this paper to address the extensions in detail, some context is provided to show in which areas the model can be expanded.

\subsubsection{Multi-mode project scheduling}
In multi-mode RCPS, activities can be executed in multiple modes. The activity modes are characterized by a certain resource requirement and a duration. For instance, one can either build a wall with one worker in three days or with two workers in two days. Instead of letting $\dur{v_i}$ be the duration of an activity $v_i$, the duration is dependent on the mode in in which the activity is executed. The notation therefore becomes $\textit{Dur}(v_i, m)$ and $\textit{Av}(v_i, m)$ in which $m$ is the mode in which the activity is executed. See \cite{herroelen05} for details. The approach to scheduling that is discussed in this paper does not account for multi-mode activities.

\subsubsection{Stochastic processing times}
Oftentimes it will be hard to estimate processing times for activities in practice and there is a big chance that the actual processing time will be different from the fixed estimate that was used in the model. Instead of using deterministic values for the activity duration, one could introduce a stochastic variable which has a certain probability distribution. In \cite{brucker99}, it is argued that stochastic processing times for project scheduling are an important step closer towards the real-life scenario. In this paper, we will focus on fixed activity durations.
\end{comment}

\newpage

\section{Schedule construction using PCP}
\label{section:PCP}
Since the RCPS problem belongs to the class of \emph{NP-complete} problems, no algorithm capable of constructing an optimal solution in $O(n^k)$ is known (unless of course, $P=NP$). In order to devise a solution using polynomial time, an approximation algorithm needs the be used. In this section, the approach of Precedence Constraint Posting (PCP) will be discussed.

\subsection{Introduction}
When regarding the RCPS problem, the problem has got both a time-related as well as a resource-related aspect. 
The time-related aspect focusses on scheduling the activities with respect to the given precedence constraints. 
As will be discussed below, this time-aspect of the problem can be reduced to a \emph{Simple Temporal Problem} (STP) \cite{lombardi10} which can be solved in $O(n^3)$ time. 
The result of this STP is a schedule or a set of schedules for which needs to be verified if the resource constraints are met. 
If there are any resource violations, \emph{Precedence Constraint Posting} (PCP) will offer a method of adding constraints to the STP in such a way that forces the resource requirements to be met in the generated schedule. 

The two aspects of this problem are regarded as two different layers, the \emph{Time Layer} and the \emph{Resource Layer}.
The Time Layer will first be discussed along with the reduction to the aforementioned STP. 
This STP will then be used to explain the relevance of the Resource Layer.

\subsection{Time Layer}
The temporal aspect of the problems deals with activity durations and precedence constraints.
These relations can be represented as a graph $G=(V,E)$ as defined in section 2.1. 
This so-called \emph{precedence graph} can be reduced to a \emph{Simple Temporal Network} (STN), the format in which an STP is expressed.

\subsubsection{Simple Temporal Networks (STN)}
An STN consists of a set $P$ of so-called \emph{time points} with associated time windows.
Relations between each pair of time points can be specified.
Using a technique called \emph{constraint propagation} \cite{policella07}, it is possible to bound the domains of each time point $t_i$ such that $t_i \in [lb_i, ub_i]$, in which $lb_i$ represents the lower bound and $ub_i$ represents the upper bound of the time domain for the time point $t_i$, in such a way that all relations are satisfied.
It has been proven that it is possible to find such bounds for in $O(n^3)$ time \cite{policella07}.
Since a solution assigns values (instead of bounds) to each time point, in order to obtain a solution, a value needs to be given to each time point $t_i$.
One could, for example, take the lower bound of each time point and use that as the value for each time point.

Relations between time points can be added in the following way:
\begin{itemize}
\item $t_i \xrightarrow{[d_{ij},D_{ij}]} t_j$ called a \emph{free constraint}, meaning that the time $\delta$ between the two time points $t_i$ and $t_j$ is allowed to be scheduled within the range of $[d_{ij},D_{ij}]$.
\end{itemize}

\subsubsection{Reduction to STN}
\begin{enumerate}
\item Let S be an empty graph
\item For each activity $v_i \in V$, add two nodes (time points) $s_i$ and $e_i$ to S
\item Connect each pair of points $s_i$ and $e_i$ with a free constraint with bounds $[\dur{v_i}, \dur{v_i}]$
\item For each precedence constraint $(v_i, v_j) \in E$, add a free constraint with bounds $[0, \infty]$
\end{enumerate}

The constraint introduced in step $3$ forces the duration of the activity to be exactly $\dur{v_i}$. The constraints added in step $4$ can have any, non-negative value which enforces the activities to take place some time after each other: the definition of a precedence constraint.

Since a solution assigns values (instead of bounds) to each time point, in order to obtain a solution, a value needs to be given to each time point $t_i$.
One could, for example, take the lower bound of each time point and use that as the value for each time point in order to obtain an actual schedule instance.

\subsection{Resource Layer}
The STN that has been generated in the Time Layer provides a set of schedules in which the activities are scheduled in such a way that the given precedence constraints are satisfied. The Resource Layer focusses on the other aspect of the problem; the resources. 

\subsubsection{Resource leveling}
The Resource Layer takes the resulting STN of the Time Layer and analyzes the resource usage over time using a \emph{resource profiler}. This profiler will indicate the periods of resource violation, called \emph{contention peaks}, and the activities that create that contention peak. When the contention peaks are identified, additional precedence constraints are added to the Time Layer in order to level the resource usage and solve one or more of the detected contention peaks. The Time Layer then provides a new temporal graph (STN) which will then be analyzed again in the Resource Layer. This process continues until all resource constraints are satisfied or if precedence constraints are added in the Time Layer resulting in an inconsistent STN. 

\subsubsection{Earliest Start Time Algorithm (ESTA)}
The \emph{Earliest Start Time Algorithm} (ESTA) computes the resource profile according to one temporal solution (remember that an STN represents a set of schedules). This temporal solution is obtained by assigning to each time point in the STN to the lowest possible value in its associated time window. Again, additional constraints are posted in the Time Layer and ESTA computes a new schedule. The strength of this approach is that it shows exactly what will happen to the activities and the resource usage in the schedule. The major drawback is that this approach only guarantees the resource constraints to be met in precisely one schedule, the earliest start time solution. 

\subsubsection{Conflict Free Solution Algorithm (CSFA)}
In continuous planning, one is not so much interested in the earliest completion time, but rather in the actual robustness and flexibility of the schedule. Since the ESTA only provides one schedule, this does not give much flexibility if ons of there are disturbances in the activities (unexpected extended durations for example). Using the CSFA described in \cite{cesta98}, a schedule can be made which satisfies both the temporal constraints and the resource constraints for each combination of starting times for the activities.

\subsection{Conclusion}
Precedence Constraint Posting relies on the decoupling of the original RCPS problem into a Time Layer and a Resource Layer. By first satisfying all temporal constraints in the Time Layer before focussing on the resource constraints in the Resource Layer, a feasible solution for the problem can be found in polynomial time. This solution is a set of activities where for all activities a timespan is given in which the activity can be executed so that regardless of the combination of other start times, resource and temporal constraints are satisfied.

\newpage

\section{Continuous scheduling}
\label{section:continuous}
Resource constraint scheduling algorithms focus on solving a given set of tasks, time and resource constraints in a feasible schedule.
These solutions are effective on complex, closed sets of tasks, but the created schedules are not adaptive when the resources or set change during execution.
In many practical environments of where scheduling is performed, flexibility or robustness of the schedule is important, whilst changes occur regularly.
Either resources can be delayed, become unavailable or new tasks can be added to the set, all during execution.
In these applications the set of resources and tasks is not closed, but changes constantly.
This requires the schedule to be changed or updated during execution of the schedule.
Here we will take a deeper look at how to make resource constraint scheduling more practically applicable in continuously changing environments.

We will first discuss schedule repair solutions, which reschedule the solution when problems occur that disrupt the execution.
These repair solutions provide a way to repair a schedule when certain changes occur, for example due to delays in the process.
Then we will review flexible schedules and a special example of these flexible schedules, namely partial order schedules.
The flexible schedule contains several feasible schedules, enabling it to be adjusted to changes without the need for complete rescheduling.
Partial order schedules have an extra property, where the original output schedule can be used as input schedule for recalculating the solution.
Finally we will take a look at how precedence constraint posting can be used in combination with partial order schedules, to provide with flexible resource constraint schedules.

\subsection{Schedule repair}
One solution to handle continuous changes to the problem set is schedule repair.
This reactive solution reschedules the solution as soon as problems or conflicts are detected.
Using this type of repair requires the execution of the schedule to be paused, a new schedule to be calculated based on the changed problem set and the execution to be restarted with the new schedule.
Pausing and repairing a schedule is a very time consuming solution, especially in environments where many changes may occur.
To maximize effectiveness, the repair needs to be fast and allow the execution to continue with the minimum delay.
Different solutions exist, which mostly favor either a quick repair or a more complete rescheduling.
\cite{policella07}
Complete rescheduling often provides with a better schedule, which has a shorter execution time or uses resources more efficient.
Whilst the quick repair does not do this, but delivers a new schedule in less time.

Schedule repair gives the ability to build a schedule although there are unknown factors in the environment, but requires the execution to be paused and repaired when problems occur.

\subsection{Flexible schedules}
Another solution is a more proactive approach, using flexible schedules.
Flexible schedules are schedules which retain flexibility by coping with several changes without the need for rescheduling.
These solutions present a form of redundancy, in time and/or resources.
This redundancy represents multiple feasible schedules within one solution.
Therefore allowing the process to select the execution which most resembles the actual environment.
For flexible schedules it is required to know all sorts of changes that may occur in advance, in order to construct the correct flexible schedule.
This requirement poses a big strain on the scheduling on forehand, because several, possibly unknown, factors need to be taken into account.
If an unexpected change should occur, the schedule will most likely not be capable of handling that change.
In this case complete rescheduling is necessary.

Flexible scheduling provide a way to use the original flexible schedule and adopt it to the environment, but can only cope with known changes and problems.

\subsection{Partial order schedules}
Partial order schedules are flexible schedules which take the flexibility a step further.
In partial order schedules, as defined in \citet{policella07}, each activity retains a set of feasible start times.
They are represented by a temporal network, this network represents the different feasible schedules, similar to other flexible schedules.
In all flexible schedules there is a possibility that the changes that occur are too large to adapt the schedule and rescheduling is required.
This is where the main advantage of partial partial order schedules is, the original schedule can be used as input for the rescheduling algorithm.
By reusing the previous costly time can be preserved in building a new schedule.

The rescheduling algorithm uses a temporal network as input, which is similar to the output itself generates.
Because the original schedule is modified and reused, partial order schedules make the complete rescheduling less intensive.
This property makes partial order schedules not only flexible and robust, but also quick to repair when bigger changes occur.
Therefore partial order scheduling is especially suitable for applications in a dynamic environment.

\subsection{Precedence constraint posting in partial order schedules}
Precedence constraint posting, as discussed in the previous chapter, is a technique for finding feasible schedules by adding new precedence constraints to a temporal network.
This posting of new precedence constraints can be used to update the temporal network of a partial order schedule.
In this manner several large adjustments can be posted to a schedule, which can then be easily rescheduled.
Precedence constraint posting in combination with partial order schedules give a very robust and flexible solution to the scheduling of tasks in a dynamic environment.

\newpage

\section{Conclusion}
In this paper we discussed several aspects of the resource constraint scheduling problem.
The resource constraint scheduling problem proves to be a challenging problem in many day-to-day applications.
These applications often operate in dynamic environments, which pose many external constraints on the schedules.
To cope with these constraints and problems, some techniques are applied to provide with a more flexible and robust schedule.
Flexibility is key in providing a schedule that can function in a dynamic environment, handling the uncertainties which may occur.

At first we discussed scheduling methods and relevant techniques, to discover that not all of these schedule deliver the same flexibility as needed.
Other techniques like precedence constraint posting and temporal networks were introduced, to provide us with the tools to create more flexibility.

We then discussed several ways flexible schedules can be created and what kind of techniques can be used to do so.
Finally we found that a combination of a flexible schedule in the form of a temporal network and techniques used in partial order scheduling, provide with a very flexible structure.
These flexible schedules are capable of handling the dynamic environment better than the simpler or more rigid variants.

The capability to handle changes in the environment as well as a growing set of tasks makes this combination a good and efficient solution to many practical scheduling applications.
When scheduling becomes a big and complex part of the daily business, one might consider using precedence constraint posting to help build resource constraint schedules in a dynamic environment.


\newpage
\bibliographystyle{plainnat}
\bibliography{references}

\end{document}
