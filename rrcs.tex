\documentclass{article}
\usepackage{a4wide}
\usepackage{amssymb}
\usepackage{amsmath}
\usepackage[numbers]{natbib}
\newcommand{\leading}[1]{\textbf{#1}}
\usepackage{xcolor}

%% if your are not using LaTeX2e use instead
%% \documentstyle[bnaic]{article}

%% begin document with title, author and affiliations

\title{Precedence Constraint Posting in Continuous Planning; Practical Adoption of the Scheduling Problem.}
\author{M. van Gelderen  \and
    R.M. de Lange \and
    B. Gris\`el \and
    F. van Tienen}

\pagestyle{empty}

\newcommand{\TODO}[1]{{\color{red}\textbf{TODO: #1}}}

\begin{document}
\maketitle
\thispagestyle{empty}

\begin{abstract}
\TODO{abstract als laatste schrijven}
\end{abstract}

\section{Introduction}

The resource scheduling problem is a problem concerned with the scheduling of tasks on a specific set of resources.
These, often scarce, resources form a constraint on the volume of tasks which can be processed at any one time.
Scheduling focuses on optimizing a schedule to either consume the least amount of resources, or perform all tasks in minimum required time.  \cite{brucker99}
There are several different forms of resource constrained scheduling, all dealing with a specific type of scheduling problem.

The main aspects that will be covered in this paper are precedence constraint posting and continuous planning.
Precedence constraints require a set of tasks to be performed in a specific order, requiring one task to be performed before the other.
Also, temporal constraints have a part in this scheduling problem, but this will not be the focus in this paper.
Continuous planning is a form of scheduling where the scheduling process is not completed, the process continuous while schedule is being executed.
In this case the set of tasks to be performed is not closed, meaning that during execution of the schedule the set of tasks may change and with it the optimal solution.
However, it may also be undesirable to introduce major changes in the baseline schedule.
Therefore, the continuous planning aspect of scheduling is one of the more complex issues and also a quite common phenomenon in practical situations.

In this paper we will combine these to elements in the scheduling problem and study this in the light of robust resource constrained scheduling.
Robust resource constrained scheduling requires that the schedule not only gives an efficient solution to perform all tasks, it should also take in to account external changes.
In this paper we will use the same definition of robust as in \cite{policella07} : ``{(1) the ability to absorb exogenous and/or unforeseen events without loss of consistency, and (2) the ability to keep the pace with the execution guaranteeing a prompt answer to the various events.}''
These kinds of changes can also occur due to the continuous nature of the application in which the scheduling is applied, for example by adding new tasks to the set.

In the following chapter we will be dealing with the theory on scheduling problems and discus several relevant research papers that cover the topic.
After which the two main aspects, precedence constraint posting and continuous planning, will be discussed further and then combined and linked to practical problems.
We will take a deeper look at the practical applications for the theory, with their respective overlap and conflicts with the aforementioned theory.
As one might expect, several issues occur in day-to-day applications of scheduling and not every aspect that is found in practice is covered by theoretical research on this subject.
As is not unusual for mathematical or logical models of real-life problems, there is aways some assumption on preconditions which needs to be done to make a simplified model.

This paper is intended to link the practical applications of the scheduling problem to several aspects of the theory, thereby addressing the incompleteness of the problem theory and discover what aspects need further research.


\newpage

\section{Problem background}

Different aspects of the underlying theory leading to RRCS.  Per point explanation of the theory with references to relevant papers. This will be the main section of the paper.

\TODO{Describe what notation conventions we will adopt. I suggest using the notation from P. Brucker et al, Eauropean Journal of Operational Research 112 (1999). }

\subsection{Formal Definition}
The general definition for the resource constraint scheduling problem is defined by a tuple $(T, V, E, R)$: \cite{brucker99}

Given:
\begin{itemize}
\item a horizon T that denotes the total time horizon for the problem
\item a set of activities(jobs) $V = v_1 \ldots v_n$, with a unique dummy beginning activity $v_0$ and a unique dummy terminating activity $v_{n+1}$
\item the durations, processing time of the activities $Dur(v_i)$, so that $Dur(v_k)$ denotes the duration of activity $v_k$
\item a set of constraints which needs to be satisfied $E$, such that $(v_i,v_j) \in E$ means that $v_i$ precedes activity $v_j$. We asume that the grpah doesn't conain a cycle, otherwise the precedence relations are inconsistent.
\item a set of available resources $\mathcal{R}$, splitted in consumable(renewable) resources $\mathcal{R}^{\rho}$ and nonrenewable resources $\mathcal{R}^V$ but with a possible overlap of doubly constrained resources
\item the availabilities of the resources $Av(\mathcal{R}_i)$, so that $Av(\mathcal{R}_k)$ denotes the availability of resource $\mathcal{R}_k$
\item the per period usage for an activity $v_j$ of resource $\mathcal{R}_k$ are denoted by $Usage(v_j, \mathcal{R}_k)$, such that it represents the amount of resource $\mathcal{R}_k$ used per time period of executing activity $v_j$
\end{itemize}

Find the optimal schedule where each activity gets executed with the available resources and all the constraints between the activities are satisfied. Where $Start(v_j)$ denotes the start time of activity $v_j$ and $Stop(v_j)$ denotes the completion time of activity $v_j$. $S = (Start(v_1) \ldots Start(v_n))$ is a schedule and $C = (Stop(v_1) \ldots Stop(v_n))$ is a vector of completion times for the activities $V = (v_1 \ldots v_n)$. \\
$S_T$ defines the set if time-feasible schedules and $S_R$ defines the set of resource-feasible schedules. Thereby the conjunction of the sets $S_R$ and $S_T$ will give you the set of feasible schedules($S = S_R \cap S_T$)

\subsection{Prerequisites}
In this section, the general problem focussed on in this paper is explained. Projects composed of activities, resources and constraints are introduced and a unifying notation is presented. First, some terminology is discussed followed by a problem definition.

\subsubsection{Activities}
Each project consists of a finite set of activities(also called jobs) denoted as $V_i$. Each activity $V_i$ has a duration(processing time) $Dur(V_i) \in \mathbb{N}$; the time needed to complete the activity. The set of all activities is $V = (v_i \ldots v_n)$, where $n$ is the number of real activities, which has a unique dummy beginning activity $V_0$ and a unique dummy terminating activity $V_{n+1}$. The starting and end times of an activity, when represented in a schedule, are denoted by $Start(v_j)$ and $Stop(v_j)$ for each activity $v_j$.

\subsubsection{Resources}
An activity requires a certain amount of resources to be available upon execution. For example, changing a tyre might require two workman and one car lift. In general, there are two kinds of resources: renewable and consumable(also named nonrenewable) resources. With a consumable resources (such as petrol), once a part of a resource is used, it will no longer be available. This contrasts with renewable resources, which do not deplete when used. For renewable resources, the amount of units available of resource $\mathcal{R}_k$ at any given time is $Av(\mathcal{R}^{\rho}_k)$ (for example the number of machines in a shop). The total amount of units available for a nonrenewable resource $k$ is noted by $Av(\mathcal{R}^{V}_k)$. The set of all renewable resources within a project is indicated with $\mathcal{R}^{\rho}$ and the set of renewable resources is indicated by $\mathcal{R}^{V}$. 

Activities consume a certain amount of one or more resources. The amount that an activity $j$ consumes of a resource $k$ during a unit of time is denoted by $Usage(v_j, \mathcal{R}^{\rho}_k)$ for a renewable resource and   $Usage(v_j, \mathcal{R}^V_k)$ for an nonrenewable resource. 

\subsubsection{Contraints}
Apart from activities that consume resources, additional constraints can be added. Two of these are precedence constraints and temporal constrains. Some activities need to be executed before another activity can be executed (for example, a surface needs to be sanded before it can be painted). Such activity precedence can be visualized using a so-called precedence graph $G = (V, E)$ in which every activity is represented by a node and directed edges between nodes $v_i$ and $v_j$ are added if an action $v_i$ needs to be executed before $v_j$. This is denoted by $(v_i,v_j) \in E$, where $v_i, v_k \in V$.

Temporal constraints can be used to specify a minimum and maximum time lag between two activities $i$ and $j$. The notation used in this paper is $MinLag(v_i, v_j)$ and $MaxLag(v_i, v_j)$ respectively the minimum and the maximum time lag.

\subsubsection{Schedule}
The general idea of resource scheduling is to minimize the project execution time without over-utilising the resources or violate project constraints like the constraints discussed above. The project itself can also have a deadline, the total amount of time units before the project needs to be completed. A mapping of a unit of time in the model and an actual timespan in the real world can be indicated by the Time Horizon $t=1,2,3,\ldots,T$ and corresponding time intervals $[t-1,t]$. The genereated schedule may not exeed the Time Horizon $T$, when the project itself has an deadline.

The RCS problem schedule solution is denoted by a schedule that consists of a vector of the starting times for all the activities $S =  (Start(v_1) \ldots Start(v_n))$. There are two types of schedules, time-feasible schedules that fullfills all time constraints denoted as $S_T$ and resource-feasible schedules that fullfills all the resource constraints denoted by $S_R$. The conjunction of the time-feasible and the resource-feasible schedule will give you a solution for the RCS problem, this is denoted by $S = S_R \cap S_T$. We also define the completion times for the activities in $V$, this is denoted by $C = Stop(v_1) \ldots Stop(v_n)$ where $Stop(v_j)$ for each activity $v_j$ has to be less than the Time horizon $T$, this can be denoted as following $\forall_{v_j}(v_j < T)$.

\subsection{Problem Complexity}
The RCS problem belongs to the class of problems that are \emph{NP-hard}. According to the complexity theory a problem belongs to the class \emph{NP-hard} if its decision version is \emph{NP-complete}. That the decision version of the RCS problem belongs to the class \emph{NP-complete} is proven and thereby is proven that the RCS problem belongs to the class \emph{NP-hard}. \\
The time and space complexity of the RCS problem depends on the used algorithms and approaches used to solve the problem. There are currently two type of algorithms which are used in most cases, the exact algorithms and the approximation algorithms.
\TODO{Explain the time and space compelxity between the two different approaches}

\subsection{Problem Extension}
The basic PS problem will sometimes not suffice. Several extensions to the PS problem have been made of which some will be discussed here. 

\subsubsection{Multi-mode project scheduling}
Multi-mode project scheduling is an extension to the original PS problem. In short it takes multiple ways to perform and activity into account. The different 'modes' in which an activity can be performed have different processing times and resource requirements for that activity. See \citet{herroelen05} for details. The approach to scheduling that is discussed in this paper does not account for multi-mode activities. Research into multi-mode scheduling with PCP for continuous planning has not been done to the best of the authors' knowledge. 

\subsubsection{Stochastic processing times}
Often times it will be hard to estimate processing times for activities in practice. There is a big chance that the actual processing time will be  different from the estimate. In depth research has been done on stochastic processing time for project scheduling to model this deviation as has been shown in \cite{brucker99} covering several important problems introduced by using a stochastic distribution for activity processing times. 

\subsection{Summary}
\leading{Basic notation overview} \\
\begin{tabular}{l l}
\hline
$T$ & time horizon \\
$t$ & index of a time period in the time horizon $T = (t_1 \ldots t_n)$\\
$V$ & set of activities $V = (v_1 \ldots v_n)$ \\
$n$ & number of real activities \\
$E$ & set of constraints \\
$\mathcal{R}^{\rho}$ & consumable resources \\
$\mathcal{R}^V$ & renewable resources \\
$Dur(v_i)$ & the durability of activity $v_i$ \\
$Av(\mathcal{R}_i)$ & the availability of resource $\mathcal{R}_i$ \\
$Usage(v_j, \mathcal{R}_k)$ & the per periode usage for activity $v_j$ of resource $\mathcal{R}_k$ \\
$Start(v_i)$ & the start time of activity $v_i$ \\
$Stop(v_i)$ & the start time of activity $v_i$ \\
$MinLag(v_i, v_j)$ & the minimum time lag between activity $v_i$ and $v_k$ \\
$MaxLag(v_i, v_j)$ & the maximum time lag between activity $v_i$ and $v_k$ \\
$S$ & a schedule consisting of strart times for the activities $S = (Start(v_1) \ldots Start(v_n))$ \\
\hline
\end{tabular}




\section{Schedule construction using PCP}
A RCSP can be broken down into two parts.  The other part of the problem concerns the usage of resources by the activities. 

The general idea of finding a feasible schedule is an iterative approach that schedules the activities according to some scheduling policy (for example an earliest start policy: start every activity as soon as possible) and then checks if the resource constraints are met. 

\subsection{Time Layer}
Firstly the temporal aspect, dealing with activity durations, constraints between pairs of activities, due dates, release time and other time-related constrains. These relations can be represented as a graph $G={V,E}$, in which each vertex represents an activity and an edge connecting two activities represents a temporal constraint between them.

\subsubsection{Conversion to Temporal Constraints}

\subsubsection{Simple Temporal Networks with Uncertainty}
A Simple Temporal Network (STN) is used to represent the temporal aspect of the RCSP. An STN consists of a set $T$ of so-called \emph{time points} with associated time windows. Relations between each pair of time points can be specified. The goal of an STN is to provide each time point with a value within its time window in such a way that the relations are satisfied. 

There are two types of relations:
\begin{enumerate}
\item $t_i \xrightarrow{[d_{ij},D_{ij}]} t_j$ called a \emph{free constraint}, meaning that the time $\delta$ between the two time points $t_i$ and $t_j$ is allowed to be scheduled within the range of $[d_{ij},D_{ij}]$.
\item $t_i \xrightarrow{[d_{ij}:D_{ij}]} t_j$ called a \emph{contingent constraint}, meaning that the time $\delta$ between the two time points $t_i$ and $t_j$ lies within the range $[d_{ij},D_{ij}]$. Unlike free constraints, its exact value is not determined by the scheduler but is controlled by Nature, it is only known that the value lies within the interval $[d_{ij},D_{ij}]$. The result is that the schedule must be able to deal with all the possible values of $\delta \in [d_{ij},D_{ij}]$.
\end{enumerate}

\section{Schedule construction using PCP}
\subsection{Time Layer}
\subsubsection{Relations as Temporal Constraints}
\subsubsection{Simple Temporal Network}
Input to second layer

\subsection{Resource Layer}
\subsubsection{Resource leveling}
\subsubsection{Minimum Conflict Sets}

\subsection{Precedence Constraint Posting}
\subsubsection{Scheduling Policy}
\subsubsection{Partial Ordering Schedules}

Time Layer
-Precedence relations
Resource Layer
-Resource constraints are met
-Minimal Conflict Set
	-Definition
-Detecting MCS
	-Execution policy (earliest start policy)
	-STN

\subsection{Precedence constraint posting}
Flexible because no set start times but added constraints.

\subsubsection{Introduction}
In the classical RCPS problem described in the previous section, the input parameters of the problem, such as processing times of activities, need to be known before generating the schedule. The solution of this problem is a fixed schedule which can then be employed in practice. However, this off-line generation of the schedule does not offer the amount of flexibility that is often desired in practical problems. For example, the processing time of the activities might not be precisely known beforehand and only best-case and a worst-case times can be given. A schedule that is able to react to uncertainties and can be updated once more and more parameters become known is desired. Using Precedence Constraint Posting (PCP), the solution of the problem is a flexible augmented schedule that can be generated using an estimate of the input parameters and can then be updated in real-time (also called on-line) to adapt to the actual values.

\subsubsection{Problem Extension}
Instead of assuming each activity to have a fixed processing time, each activity is linked to a worst-case or lower bound ($p_{j} \in \mathbb{N}$) and a best-case or upper bound ($P_{j} \in \mathbb{N}$) processing time. Whereas the classical solution of the RCPS problem was given in the form of one schedule, the result of PCP is an augmented precedence graph that defines a set of possible schedules. This graph can be used to devise a starting time for activity $a_i$ in real-time, based on the execution history so far. The graph needs to be constructed in such a way that 


\section{Problem Practice}
This section will discuss how the practice problems can be dealt with using the theory described in the previous section. First, the problems that can be solved using this theory are discussed. After that, the problems which cannot be solved directly using the theory are explained.\\

\begin{itemize}
\item Project scheduling under uncertainty: Survey
and research potentials
\item A Comparison of Heuristic and Optimum Solutions
\item Reactive scheduling, improving the robustness of schedules
\end{itemize}

\subsection{Solvable problems with PCP}
\TODO{Freek}
\begin{itemize}
\item What are the examples from the real world? 
\item What are the underlying problems of these examples?
\item How can they be solved using the theory described above?
\end{itemize}

\subsection{Related problems with PCP}
\TODO{Bastiaan}
\begin{itemize}
\item What are the examples from the real world?
\item Trains, trucks, maintenance scheduling, etc.
\item Airport planning / different factors: maintenance / tanking / bagage / boarding
\item What are the underlying problems of these examples?
\item How does the above theory needs to be augmented in order to be able to solve the problems?
\end{itemize}


\section{Continuous planning} \TODO{Mick de Lange}

In this chapter continuous planning will be discussed.
Continuous planning is a much seen obstacle in practical scheduling problems.
Flexibility and robustness of a schedule are key to continuous planning, as problems or changes may occur during execution.
Several different aspects and approaches of continuous planning will be viewed in the light of the scheduling problem and related to flexibility and robustness.

The subject of continuous planning is somewhat neglected in classical scheduling problems and research.
With continuous planning one needs to take into account that the set of tasks is not complete or closed when a schedule is requested.
Changes may occur for example due to unexpected shortage of resources, delays or newly added tasks.
Therefore continuous planning requires flexibility and robustness in the solution and options of editing the solution.
In continuous planning efforts are made to provide that flexibility, whilst compromising on the optimal character of the schedule.

Resource constraint scheduling algorithms focus on solving a given set of tasks, time and resource constraints in an optimal schedule.
This makes these solutions quite effective on more complex closed sets of tasks, but the created schedules are not adaptive when the resources or set change during execution.
In current continuous planning, the solution is more flexible and adaptive, but does not necessarily represent an optimal solution with regards to extra constraints or complications.
\cite{laborie03}

\subsection{Automated planning}

A classical approach to scheduling a continuously growing set of tasks is to simply plan ahead until a set horizon.
After this horizon, the next set of tasks is planned.
This way batch after batch of tasks is scheduled, resulting in a schedule, which is most likely to be suboptimal.
The result does not allow for changes in the already planned parts and therefore possibly misses more efficient solutions and lacks adaptivity.
\cite{chien00}

In several different fields we see the search for better optimization of continuous planning solutions.
In the field of robotics, like robotic space exploration, automated planning is the main scheduling approach.
This form of AI planning focuses on realization of strategies and short-term task selection.
With this planning form autonomous robots select the tasks with only a short foresight.
In this field of AI efforts are being made to optimize the schedules, with the use of several parts of the resource constraint scheduling domain.
\cite{smith00}

The complexity of the tasks to be selected by the robots are becoming more complex, because an increasing set of variables needs to be taken in to account.
Due to these extra variables the planning problem resembles the more complex resource constraint scheduling problem.
Automated planning does not have the capabilities to take these extra constraints into consideration.
However, the solution is not to simply use resource constraint scheduling, for this does not currently give the same speed en flexibility as automated planning.
The main reason for this is that tasks need to be selected "on-the-go", thus decisions need to be made in a short timespan.
The time it takes to calculate an optimal solution would be of such length, that the set of current tasks might deviate from the selected tasks in the schedule.

\subsection{Iterative repair}
Iterative repair is the process of repeatedly repairing the schedule when required.
This requires that there is a schedule at first, which then can be mended on several occasions to account for the changes in the problem set.
The iterative solution also needs to be fast, for it should be run multiple times during the execution process, all the while not slowing the process down.

The iterative process works as follows.
Once a change to either the state or goals of the current plan are detected, the iterative solution should identify conflicts and predict the effects of the changes to the plan.
Repairs are made to the plan, preventing conflicts and searching for more optimal solution.
The plan is repaired with regard to the least amount of delay in the execution process.
The portion of the old plan, which was not yet executed might have changed completely due to a new task with higher priority or possibly a better order to perform the new set of jobs in.
\cite{chien00}

In fact iterative repair is a solution which extends the current scheduling problems, by adding repairing algorithms.
This way the original non-flexible schedule can be used in a more flexible manner, but it is has not become more flexible.
In other words, iterative repair provides a method of making planning more flexible, although not really making a flexible schedule.
Iterative repair is a solution which can be very labor intensive, because of the repeated analyzing and rebuilding of the schedule.

\subsection{Partial order schedules}
Partial order schedules are solutions which take the flexibility a step further.
In partial order schedules, as defined in \cite{policella07}, each activity retains a set of feasible start times.
In this manner a set of possible schedules is available, rather than just one solution.
In case there is a change, which is not too big, a simple modification to the schedule suffices to generate a new feasible solution.
Partial order scheduling uses repairs, just as the iterative repair schedules that were previously mentioned.
However, the partial order schedule already contains several options for an easy fix of the schedule.
Therefore the time consumed in the recalculating of the schedule is reduced tremendously.

There is a possibility of the changes being significant enough to force a complete rescheduling, which takes more time than the quick repair.
The partial order schedules are quicker to repair, but require more time initially and when large changes occur.
In addition to this problem, the quick repairs do not represent an optimal schedule.
Therefore partial order schedules are a better and more flexible manner of continuous planning, but still does not result in the best solution. 


\subsection{Precedence Contraint Posting in Continuous planning}
\TODO{Mick van Gelderen}
Precedence Constraint Posting (PCP) is a technique for finding feasible schedules by adding new precedence constraints to the original project graph in such a way that start times can be computed for whatever combination of activity durations without compromising the resource constraints. We will first take a look at how PCP works and then describe how this technique can be utilized in continuous planning. 

The PCP approach to scheduling is suitable for continuous planning. A lot of the methods for PCP create a POS in the form of Simple Temporal Network. This network has temporal constraints that ensure not only consistency of the activity precedence relations but also that resource capacities are not exceeded. An advantage of using STNs is that it is possible to post additional constraints during schedule executions to recalculate a new feasible schedule from the POS in polynomial time.

If for example a project manager notices that a certain activity takes longer than expected, the scheduler can post an additional temporal constraint to the POS and try to create a new schedule from it. Notice that the new schedule will probably not be optimal and in some cases it might not be possible to create a new resource and time feasible schedule. In that case a new POS has to be created from the remaining project graph. 

A complete recalculation will not be necessary as frequent as with other techniques because the POS can be utilized in some cases of, possibly unexpected, lowered resource capacity, extended activity processing time or additional activities. 

\section{Conclusion}
\TODO{The overall conclusion. Huidige tekst verplaatst uit continuous}
Continuous planning poses a special problem in the scheduling domain.
Although continuous planning seems obvious and the presence is ubiquitous in practical applications, there is no effective theoretical solution capable of providing a optimal solution in continuous planning.
There is always a trade-of between optimal solutions and processing speed.
Often the pressure to provide a quick solution overrules the optimum option.

\bibliographystyle{plainnat}
\bibliography{references}

\end{document}








