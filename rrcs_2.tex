\documentclass{article}
\usepackage{a4wide}
\usepackage{amssymb}
\usepackage{amsmath}
\usepackage[numbers]{natbib}
\usepackage{xcolor}
\usepackage[colorlinks,linkcolor=blue,urlcolor=blue,citecolor=blue]{hyperref}
\usepackage{mathrsfs}
\usepackage{comment}
\usepackage{tabularx}
\usepackage{booktabs}
\usepackage{caption}% http://ctan.org/pkg/caption
\captionsetup[table]{justification=raggedright, singlelinecheck=off}

\usepackage{tikz}
\usetikzlibrary{calc,positioning,shapes,arrows,decorations.pathreplacing}
\usetikzlibrary{shapes.multipart}


\title{Resource Constrained Project Scheduling for continuous applications using Precedence Constraint Posting.}
\author{M. van Gelderen  \and
    R.M. de Lange \and
    B. Gris\`el \and
    F. van Tienen}
\date{}

\pagestyle{empty}

\newcommand{\TODO}[1]{{\color{red}\textbf{TODO: #1}}}

\newcommand{\res}[0]{\ensuremath{R}} %resources
\newcommand{\av}[1]{\ensuremath{av(r_{#1})}} %availability
\newcommand{\capa}[1]{\ensuremath{cap(r_{#1})}} %capacity
\newcommand{\dur}[1]{\ensuremath{dur(v_{#1})}} %durability
\newcommand{\usage}[2]{\ensuremath{usage(v_{#1}, r_{#2})}} %usage of resource #2 by activity #1
\newcommand{\start}[1]{\ensuremath{start(v_{#1})}} %start time
\newcommand{\makespan}[1]{\ensuremath{C_{max}(#1)}} %makespan

\newenvironment{definition}[1][Definition]{\begin{trivlist}
\item[\hskip \labelsep {\bfseries #1}]}{\end{trivlist}}

\setlength{\parindent}{0pt}
\setlength{\parskip}{\baselineskip}

\begin{document}
\maketitle
\thispagestyle{empty}

\begin{abstract}
\TODO{rewrite when paper finished}
\end{abstract}


\newpage


\section{Scope and purpose}

\TODO{Roelf}
%		-Motivation, waarom probleem behandelen?
%		-Wat staat waar?
%		-Dat het probleem moeilijk is dat we er in hoofdstuk 2 op terug komen?


\newpage


\section{Problem background}

\TODO{Roelf: intro}
%		-What are scheduling problems?
%		-Examples form real life.
%		-What is RSPCP and how can it solve them.
%		-Why imporant.
%		-Voorbeelden wat meer uitwerken. Intuitief verhaal.}

\TODO{Freek: running example}

In this paper we will use a simple running example based on a painting job to explain the RCPSP.

The problem consists of workers that need to finish three painting jobs: a door, a fence and a wall.
Because the the door and the fence are made of wood, both surfaces needs to be sanded first (\emph{precedence constraints}).
Sanding is not necessary for the wall, because the wall is made of bricks.
When the job is finished all painting brushes need to be cleaned and this can only be done when all the painting is finished.
This gives us six \emph{activities}: sanding the the door, sanding the fence,  painting the door, painting the fence, painting the wall and cleaning the painting brushes.
And the following precedence constraints: sanding the door before painting the door, sanding the fence before painting the fence and all the painting activities need to be done before cleaning the painting brushes.

Because the size of each surface in the painting jobs differs, not each activity takes the same amount of time to finish (\emph{duration}).
Since this is not the first painting job the workers have done, we assume that they know the exact duration for each activity:
\begin{itemize}
\item sanding the door: 3 hours
\item sanding the fence: 2 hours
\item painting the door: 2 hours
\item painting the fence: 1 hour
\item painting the wall: 5 hours
\item cleaning the painting brushes: 1 hour
\end{itemize}
% moet nog even goed gestyled worden en even gecontroleerd worden of tijden beetje goed zijn
Normally this would be an easy job, but they have forgotten a lot of their tools (\emph{resources}).
So now they only have two painting brushes and one sanding machine (\emph{resource constraints}), which makes it impossible to sand more than one surface at the same time and to paint more than two surfaces at the same time.
This shows us that the \emph{capacity} of painting surfaces at the same time is two and the capacity of sanding surfaces at the same time is one.

Since every sanding job uses the sanding machine as a resource, the usage of sanding the door and sanding the fence is $1$ for the sanding machine.
Also every painting job uses the painting brush as a resource, so that painting the door, painting the fence and painting the wall all have a usage of $1$ on the painting brush.
There is only one activity left, cleaning the painting brushes, which will use both of the painting brushes resource so that the usage on the painting brushes is $2$.
Because a sanding activity doesn't use a painting brush, a painting activity doesn't use a sanding machine and cleaning the painting brushes doesn't use a sanding machine all of the other usages are $0$.

The painters want to make a schedule, where they define each starting time for the activities in such a way that the resulting schedule is feasible.
This means that the sanding of each wooden surface (the door and the wall) needs to be finished before starting the painting of that surface.
Because of the limited resources, they can never paint more than two surfaces at the same time, or sand more than one surface at the same time.
There are also a lot of other jobs to do, so the painter not only want a feasible schedule but also want to optimize the time in which they finish all the activities.

$V = \{v_1, \ldots, v_6\}$\\
$E = \{(v_1, v_3), (v_2, v_4), (v_3, v_6), (v_4, v_6), (v_5, v_6)\}$\\
$R = \{r_1, r_2\}$\\
$\dur{1} = 3, \dur{2} = 2, \dur{3} = 2, \dur{4} = 1, \dur{5} = 5, \dur{6} = 1$\\
$\usage{1}{2} = 1, \usage{2}{2} = 1, \usage{3}{1} = 1, \usage{4}{1} = 1, \usage{5}{1} = 1, \usage{6}{1} = 2$\\
$\capa{1} = 2, \capa{2} = 1$

\subsubsection{Definition}
\begin{figure}[h]
	\centering
	% Author: Mick van Gelderen
% Created from http://www.texample.net/tikz/examples/class-diagram/ and other examples and resources. 

%\documentclass{standalone}
%\usepackage{tikz}
\usetikzlibrary{calc,positioning,shapes,arrows,decorations.pathreplacing}
\usetikzlibrary{shapes.multipart}


%\begin{document}

\tikzstyle{activity}=[rectangle, draw=black, text centered, text=black, thick, minimum height=1.8em, minimum width=1.8em]
\tikzstyle{dummy}=[rectangle, fill=black!7, draw=black, text centered, text=black, thick, minimum height=1.8em, minimum width=1.8em]
\tikzstyle{precedence}=[->,>=stealth, draw=black!70, thick]

\newcommand{\activity}[3]{\node (v#1) [activity, text width=#2cm, #3] {$v_#1$};}

\begin{tikzpicture}[node distance=.8cm]
		\node (v0) [dummy] {$v_0$};
    \activity{2}{2}{right=1.6cm of v0}
    \activity{1}{3}{above=of v2}
    \activity{3}{2}{right=of v1}
    \activity{4}{1}{below=of v3}
    \activity{5}{5}{below=of v2}
    \activity{6}{1}{right=1.6cm of v4}
		\node (v7) [dummy, right=of v6] {$v_7$};
		
    \draw[precedence] (v0.east) to[out=0,in=180] (v1.west);
    \draw[precedence] (v0.east) to[out=0,in=180] (v2.west);
    \draw[precedence] (v0.east) to[out=0,in=180] (v5.west);
    \draw[precedence] (v1.east) to[out=0,in=180] (v3.west);
    \draw[precedence] (v2.east) to[out=0,in=180] (v4.west);
    \draw[precedence] (v3.east) to[out=0,in=180] (v6.west);
    \draw[precedence] (v4.east) to[out=0,in=180] (v6.west);
    % start bending from the projection v4 on the line y=v3.y
    \draw[precedence] (v5.east) -- ($(v5)!(v4)!($(v5)+(1,0)$)$) to[out=0,in=180] (v6.west);
		\draw[precedence] (v6.east) to[out=0,in=180] (v7.west);		    

    % Descriptions
    \path (v3.east) to[out=0,in=180] coordinate (temp) (v6.west);
    \draw[shorten <=2pt, shorten >=2pt] (temp) -- ++(.8,.6) node[anchor=west, yshift=2pt, text width=2cm] {Precedence constraint $(v_3, v_6) \in E$};
    
    \draw[shorten <=2pt, shorten >=2pt] (v5.south) -- ++(.8,-.6) node[anchor=west, yshift=-2pt] {Activity $v_5$};
    
    % Funky braces
    \draw[decorate,decoration={brace,amplitude=8pt}]
        let \p1 = (v5.west), \p2 = (v6.east), \p3 = (v1) in 
        (\x1, \y3+1cm) -- (\x2, \y3+1cm) node[midway, above=10pt] {$V = \{v_1, \ldots, v_6\}$};
        
    \draw[decorate,decoration={brace,amplitude=8pt}]
        let \p1 = (v0.west), \p2 = (v7.east), \p3 = (v1) in 
        (\x1, \y3+2cm) -- (\x2, \y3+2cm) node[midway, above=10pt] {$W = V \cup \{v_0, v_7\}$};
    
\end{tikzpicture}

%\end{document}
	\caption{The activity graph for the running example. }
	\label{fig:activity_graph}
\end{figure}
\emph{Activities} are specified by a set $V = \{v_1, \ldots, v_n\}$.
This is a set of activities where the total amount of activities is $n \in \mathbb{N}$ and each activity is represented by a unique identifier. %identifier??
In the running example we have six activities ($n = 6$): sanding the the door $v_1$, sanding the fence $v_2$,  painting the door $v_3$, painting the fence $v_4$, painting the wall $v_5$ and cleaning the painting brushes $v_6$.
Which gives us the total set of activities $V = \{v_1, \ldots, v_6\}$ for the running example.
These activities are represented in Figure \ref{fig:activity_graph} as blocks.

The \emph{Duration} for each activity $v_i \in V$ is specified by $\dur{i} \in \mathbb{N}$.
In the running example the activities have the following duration: $\dur{1} = 3, \dur{2} = 2, \dur{3} = 2, \dur{4} = 1, \dur{5} = 5, \dur{6} = 1$.
The duration of each activity in the running example is represented by the length of the block in Figure \ref{fig:activity_graph}.

\emph{Precedence constraints} are specified by a set $E \subset V \times V$.
This means that a precedence relation between activities $v_i$ and $v_j$, where $v_j$ can be started only after $v_i$ is finished, is represented as $(v_i, v_j) \in E$.
The running example has several precedence constraints between the activities, which will give the following set $E = \{(v_1, v_3), (v_2, v_4), (v_3, v_6), (v_4, v_6), (v_5, v_6)\}$.
Figure \ref{fig:activity_graph} shows the precedence constraints for the running example, represented as arrows between the nodes.

The \emph{extended activity set} is specified by $W = V \cup \{v_0, v_{n+1}\}$.
Which consists of the activity set $V$ and incorporates a unique dummy beginning activity $v_0$ and a unique dummy ending activity $v_{n+1}$. 
These dummy activities have a duration of $0$ periods.
Precedence constraints are added to ensure that the starting activity starts before every other activity and that every activity is completed before the ending activity. 
For the running example we will have a set $W = \{v_0, \ldots v_7\}$.
Figure \ref{fig:activity_graph} shows $W$ for the running example. 

The \emph{constraints graph} is specified by a directed graph $G = (V, E)$.
It consists of the activities $V$, which are represented as nodes, and the precedence constraints $E$, which are represented as connections between nodes.
The constraints graph for the running example is represented in Figure \ref{fig:activity_graph}, where the precedence constraints are reduced to a minimum.

\emph{Resources} are specified by a set $R = \{r_1, \ldots, r_m\}$.
The total amount of resources is $m \in \mathbb{N}$ and each resource is represented by a unique identifier. %identifier??
In the running example we have two resources ($m = 2$): the sanding machine $r_1$ and the painting brushes $r_2$.
Which gives us the total set of resources $R = \{r_1, r_2\}$

\subsection{Resource constrained project scheduling problem}

\TODO{Bastiaan}
%		-Lijst met uitleg van termen naar aanleiding van running example.
%		-Duidelijk schedule definieren.
%		-Dit is een activity => Ow kut nu hebben we ook nog resources


\TODO{Freek: Fomele definitie en wat is een oplossing}

\begin{definition}
Given:
A set of activities $V = v_1, \ldots, v_n$, a set of resources $R = r_1, \ldots, r_m$ and a set of precedence constraints $E \subset V \times V$.
Where each activity $v_i \in V$ has a duration $\dur{i} \in \mathbb{N}$ and each resource $r_j \in R$ has a capacity $\capa{j} \in \mathbb{N}$. 
And each activity $v_i \in V$ can use the capacity of a resource $r_j \in R$ with usage $\usage{i}{j} \in \mathbb{N}$.

Find:
A feasible schedule $S = s_1, \ldots s_n$, consisting of starting times for the activities in $V$, where each precedence constraint in $E$ holds.
And no resource exceeds its capacity in usage at any time in the schedule.
\end{definition}


\TODO{Freek: complexiteit en oplosbaarheid.}
%		-Wat zijn de voor en nadelen van heuristic (niet optimaal, eigenlijk is het best kut maar we kunnen niet anders)}


\newpage


\section{Schedule construction} \TODO{andere titel?}

\TODO{Roelf: intro}
%		-Wat gaan we benaderen (relefantie van STN)
%		-Geen volledige reductie, dus niet alle instanties worden chill afgebeeld
%		-Globale idee geven van opzet van de reductie en hoe je daarmee het probleem kan 

\subsection{Simple temporal networks}

\TODO{Bastiaan + Mick}
%		-Wat is een STN probleem (wat is het resultaat)
%		-Complexiteit
%		-Hoe los je het op


\subsubsection{Reduction to STN}

\TODO{Bastiaan + Mick}
%		-Het is dus geen goeie/volledige reductie :(
%		-Algemene/formele reductie Reductiedefinitie


\subsection{Resource}

\TODO{Bastiaan + Mick}
%		-Uitleggen huidige STN niet werkt (alleen temporal en niet resource feasable)
%		-Globale idee van resource leveling using PCP
%		-Algoritmw voor resource leveling (volgens ESTA en aan de hand van example)
%		-Algemeen en Formeel algoritme

\subsection{Final Schedule}

\TODO{Bastiaan + Mick}
%		-Hoe maak je van dat STN weer een schedule
%		-Hoe voldoet het schedule (aan alle constraints)


\newpage


\section{Conclusion}

\TODO{Roelf}

%test

\citet{brucker99}

\citet{herroelen05}

\citet{policella07}

\citet{lombardi10}

\citet{cesta98}

\citet{deblaere10}


\newpage
\bibliographystyle{plainnat}
\bibliography{references}

\end{document}
