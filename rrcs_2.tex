\documentclass{article}
\usepackage{a4wide}
\usepackage{amssymb}
\usepackage{amsmath}
\usepackage[numbers]{natbib}
\usepackage{xcolor}
\usepackage[colorlinks,linkcolor=blue,urlcolor=blue,citecolor=blue]{hyperref}
\usepackage{mathrsfs}
\usepackage{comment}
\usepackage{tabularx}
\usepackage{booktabs}
\usepackage{caption}% http://ctan.org/pkg/caption
\captionsetup[table]{justification=raggedright, singlelinecheck=off}

\usepackage{tikz}
\usetikzlibrary{calc,positioning,shapes,arrows,decorations.pathreplacing}
\usetikzlibrary{shapes.multipart}


\title{Resource Constrained Project Scheduling for continuous applications using Precedence Constraint Posting.}
\author{M. van Gelderen  \and
    R.M. de Lange \and
    B. Gris\`el \and
    F. van Tienen}
\date{}

\pagestyle{empty}

\newcommand{\TODO}[1]{{\color{red}\textbf{TODO: #1}}}

\newcommand{\renres}[0]{\ensuremath{R^\rho}} %renewable resource
\newcommand{\conres}[0]{\ensuremath{R^\nu}} %consumable resource
\newcommand{\res}[0]{\ensuremath{R}} % all resources
\newcommand{\av}[1]{\ensuremath{Av(r_{#1})}} %availability
\newcommand{\dur}[1]{\ensuremath{Dur(v_{#1})}} %durability
\newcommand{\usage}[2]{\ensuremath{Usage(v_{#1}, r_{#2})}} %usage of resource #2 by activity #1
\newcommand{\start}[1]{\ensuremath{Start(v_{#1})}} %start time
\renewcommand{\stop}[1]{\ensuremath{Stop(v_{#1})}} %stop time
\newcommand{\makespan}[1]{\ensuremath{C_{max}(#1)}} %makespan

\setlength{\parindent}{0pt}
\setlength{\parskip}{\baselineskip}

\begin{document}
\maketitle
\thispagestyle{empty}

\begin{abstract}
\TODO{rewrite when paper finished}
\end{abstract}


\newpage


\section{Scope and purpose}

\TODO{Roelf}
%		-Motivation, waarom probleem behandelen?
%		-Wat staat waar?
%		-Dat het probleem moeilijk is dat we er in hoofdstuk 2 op terug komen?


\newpage


\section{Problem background}

\TODO{Roelf: intro}
%		-What are scheduling problems?
%		-Examples form real life.
%		-What is RSPCP and how can it solve them.
%		-Why imporant.
%		-Voorbeelden wat meer uitwerken. Intuitief verhaal.}

\TODO{Freek: running example}

\subsection{Resource constrained project scheduling problem}

\TODO{Bastiaan}
%		-Lijst met uitleg van termen naar aanleiding van running example.
%		-Duidelijk schedule definieren.
%		-Dit is een activity => Ow kut nu hebben we ook nog resources


\TODO{Freek: Fomele definitie en wat is een oplossing}


\TODO{Freek: complexiteit en oplosbaarheid.}
%		-Wat zijn de voor en nadelen van heuristic (niet optimaal, eigenlijk is het best kut maar we kunnen niet anders)}


\newpage


\section{Schedule construction} \TODO{andere titel?}

\TODO{Roelf: intro}
%		-Wat gaan we benaderen (relefantie van STN)
%		-Geen volledige reductie, dus niet alle instanties worden chill afgebeeld
%		-Globale idee geven van opzet van de reductie en hoe je daarmee het probleem kan 

\subsection{Simple temporal networks}

\TODO{Bastiaan + Mick}
%		-Wat is een STN probleem (wat is het resultaat)
%		-Complexiteit
%		-Hoe los je het op


\subsubsection{Reduction to STN}

\TODO{Bastiaan + Mick}
%		-Het is dus geen goeie/volledige reductie :(
%		-Algemene/formele reductie Reductiedefinitie


\subsection{Resource}

\TODO{Bastiaan + Mick}
%		-Uitleggen huidige STN niet werkt (alleen temporal en niet resource feasable)
%		-Globale idee van resource leveling using PCP
%		-Algoritmw voor resource leveling (volgens ESTA en aan de hand van example)
%		-Algemeen en Formeel algoritme

\subsection{Final Schedule}

\TODO{Bastiaan + Mick}
%		-Hoe maak je van dat STN weer een schedule
%		-Hoe voldoet het schedule (aan alle constraints)


\newpage


\section{Conclusion}

\TODO{Roelf}

%test

\citet{brucker99}

\citet{herroelen05}

\citet{policella07}

\citet{lombardi10}

\citet{cesta98}

\citet{deblaere10}


\newpage
\bibliographystyle{plainnat}
\bibliography{references}

\end{document}
