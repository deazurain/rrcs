% Author: Mick van Gelderen
% Created from http://www.texample.net/tikz/examples/class-diagram/ and other examples and resources. 

%\documentclass{standalone}
%\usepackage{tikz}
\usetikzlibrary{calc,positioning,shapes,arrows,decorations.pathreplacing}
\usetikzlibrary{shapes.multipart}


%\begin{document}

\tikzstyle{activity}=[rectangle, draw=black, text centered, text=black, thick, minimum height=1.8em, minimum width=1.8em]
\tikzstyle{dummy}=[rectangle, fill=black!7, draw=black, text centered, text=black, thick, minimum height=1.8em, minimum width=1.8em]
\tikzstyle{precedence}=[->,>=stealth, draw=black!70, thick]

\newcommand{\activity}[3]{\node (v#1) [activity, text width=#2cm, #3] {$v_#1$};}

\begin{tikzpicture}[node distance=.8cm]

    \node (v2) [activity] {Hek schuren};
    \node (v1) [activity, above=of v2] {Deur schuren};
    \node (v3) [activity, right=of v1] {Deur verfen};
    \node (v4) [activity, below=of v3] {Hek verfen};
    \node (v5) [activity, below=of v2] {Muren verfen};
    \node (v6) [activity, right=1.6cm of v4] {Opruimen};

    \draw[precedence] (v1.east) to[out=0,in=180] (v3.west);
    \draw[precedence] (v2.east) to[out=0,in=180] (v4.west);
    \draw[precedence] (v3.east) to[out=0,in=180] (v6.west);
    \draw[precedence] (v4.east) to[out=0,in=180] (v6.west);
    % start bending from the projection v4 on the line y=v3.y
    \draw[precedence] (v5.east) -- ($(v5)!(v4)!($(v5)+(1,0)$)$) to[out=0,in=180] (v6.west);
    
\end{tikzpicture}

%\end{document}