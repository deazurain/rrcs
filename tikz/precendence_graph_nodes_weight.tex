% Author: Mick van Gelderen
% Created from http://www.texample.net/tikz/examples/class-diagram/ and other examples and resources. 

%\documentclass{standalone}
%\usepackage{tikz}
\usetikzlibrary{calc,positioning,shapes,arrows,decorations.pathreplacing}
\usetikzlibrary{shapes.multipart}


%\begin{document}

\tikzstyle{precedence}=[->,>=stealth, draw=black!70, thick]


	\begin{tikzpicture}[
		->,
		>=stealth',
		shorten >=1pt,
		auto,
		node distance=0.8cm,
	     semithick
	  ]

    \node[state] (v2) [] {$v_2$};
    \node[state] (v1) [above=of v2] {$v_1$};
    \node[state] (v3) [right=of v1] {$v_3$};
    \node[state] (v4) [below=of v3] {$v_4$};
    \node[state] (v5) [below=of v2] {$v_5$};
    \node[state] (v6) [right=1.6cm of v4] {$v_6$};

   \draw[precedence] (v1.east) to[out=0,in=180] node [fill=white, font=\small] {$[0,\infty]$} (v3.west);
    \draw[precedence] (v2.east) to[out=0,in=180] node [fill=white, font=\small] {$[0,\infty]$} (v4.west);
    \draw[precedence] (v3.east) to[out=0,in=180] node [fill=white, font=\small] {$[0,\infty]$} (v6.west);
    \draw[precedence] (v4.east) to[out=0,in=180] node [fill=white, font=\small] {$[0,\infty]$} (v6.west);
    % start bending from the projection v4 on the line y=v3.y
    \draw[precedence] (v5.east) -- ($(v5)!(v4)!($(v5)+(1,0)$)$) to[out=0,in=180] node [fill=white] {$[0,\infty]$} (v6.west);
        
\end{tikzpicture}

%\end{document}
